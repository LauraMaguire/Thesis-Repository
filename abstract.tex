\abstract{~\OnePageChapter{}
  
  Interactions between particles and their environment can alter the
  dynamics of biological systems. In crowded media like the cell,
  interactions with obstacles can introduce anomalous subdiffusion. Active
  matter systems, \textit{e.g.}, bacterial swarms, are nonequilibrium 
  fluids where interparticle interactions and activity cause collective motion
  and dynamical phases. In this thesis, I discuss my advances in the
  fields of crowded media and active matter. For crowded media, I studied the
  effects of soft obstacles and bound mobility on tracer diffusion using a
  lattice Monte Carlo model.  I characterized how bound motion can minimize the
  effects of hindered anomalous diffusion and obstacle percolation, which has
  implications for protein movement and interactions in cells. I extended the
  analysis of binding and bound motion to study the effects of transport across
  biofilters like the nuclear pore complex (NPC). Using a minimal model, I made
  predictions on the selectivity of the NPC in terms of physical parameters.
  Finally, I looked at active matter systems.  Using dynamical density
  functional theory, I studied the temporal evolution of a self-propelled needle
  system.  I mapped out a dynamical phase diagram and discuss the connection
  between a banding instability and microscopic interactions.

}

