\abstract{~\OnePageChapter{}

Selective biofilters are essential to life, controlling the transport of proteins, nucleic acids, and other macromolecules. Of particular interest are filters that require rapid motion or high flux of proteins that must still bind targets with high specificity.  Despite the apparent competition between these two attributes, many selective materials exist which leverage binding interactions to fulfill both requirements.  In this work, we investigate the mechanisms by which such filters function through both modeling and experiment, using the nuclear pore, a well-studied example of selective transport, as inspiration.  The nuclear pore, a channel lined with intrinsically disordered FG nucleoporins, permits a high flux of transport factor proteins and their cargoes while suppressing transport of proteins that cannot bind to the FG nucleoporins.  We developed a minimal model of nuclear transport that relies on the bound-state mobility of the Nup-transport factor complex for selectivity.  This model reproduces the experimentally-observed properties of the nuclear pore and demonstrates that bound-state diffusion can arise from transient, multivalent binding and binding to flexible, dynamic tethers.  We then designed tunable hydrogel mimics of the nuclear pore for use in measuring bound-state diffusion and testing the predictions of our model.  Fluorescence microscopy demonstrated that our mimics display non-zero bound diffusion.  Both the model and experimental system are sufficiently general that their principles can be applied to a wide variety of selective biomaterials.
}

%Biopolymeric filters are essential to life.  Nuclear transport in particular is an unusual form of filtering in which the flux of select large particles is greatly enhanced over that of small nonspecific molecules. The nuclear pore complex, a channel lined with intrinsically disordered FG nucleoporins, facilitates all transport between the nucleus and cytoplasm. It prevents most macromolecules from crossing the nuclear envelope while allowing the passage of transport factors and their cargo. While the basic biochemical interactions leading to transport are well-understood, the detailed mechanism remains a topic of significant debate. We have developed a model of nuclear transport which predicts that nuclear pore selectivity is largely determined by the mobility of FG nucleoporins–transport factor complexes within the pore. We test this prediction by measuring bound-state diffusion of transport factors in tunable nuclear pore mimics which consist of hydrogels filled with FG nucleoporins. Bound-state diffusion is determined for several conditions, including Nups of varying length. Bound-state mobility occurs in many biological systems in addition to the nuclear pore complex and could help explain the selectivity of other biopolymeric filters.

%Few cellular processes require such intricate active control as transport through the nuclear envelope. The nuclear pore complex (NPC) facilitates all transport, preventing most macromolecules from crossing the envelope while allowing the passage of transport factors (TFs) and their cargo. While the basic biochemical interactions of transport are well-understood, the detailed mechanism remains a topic of significant debate. We also model transport using reaction-diffusion equations.  The results suggest that (1) the flexible nature of the disordered FG nups and (2) the transient, multivalent nature of FG nup – TF interactions are together sufficient for selectivity.  Our model makes selectivity predictions that will be directly testable in our experimental setup.  We aim to use the model to tune the mimic's parameters to maximize selectivity. We create tunable mimics of the NPC using PEG hydrogels filled with FG nucleoporins (FG nups), the intrinsically disordered proteins that line the NPC channel in vivo.

%Few cellular processes require such intricate active control as transport through the nuclear envelope. The nuclear pore complex (NPC) facilitates all transport, preventing most macromolecules from crossing the envelope while allowing the passage of transport factors (TFs) and their cargo. While the basic biochemical interactions of transport are well-understood, the detailed mechanism remains a topic of significant debate. We create tunable mimics of the NPC using PEG hydrogels filled with FG nucleoporins (FG nups), the intrinsically disordered proteins that line the NPC channel in vivo. We also model transport using reaction-diffusion equations.  The results suggest that (1) the flexible nature of the disordered FG nups and (2) the transient, multivalent nature of FG nup – TF interactions are together sufficient for selectivity.  Our model makes selectivity predictions that will be directly testable in our experimental setup.  We aim to use the model to tune the mimic's parameters to maximize selectivity.

%Few cellular processes require such intricate active control as transport through the nuclear envelope. The nuclear pore complex (NPC) facilitates all transport, preventing most macromolecules from crossing the envelope while allowing the passage of transport factors (TFs) and their cargo. While the basic biochemical interactions of transport are well-understood, the detailed mechanism remains a topic of significant debate. We create tunable mimics of the NPC using PEG hydrogels filled with FG nucleoporins (FG nups), the intrinsically disordered proteins that line the NPC channel in vivo. We also model transport using reaction-diffusion equations.  The results suggest that (1) the flexible nature of the disordered FG nups and (2) the transient, multivalent nature of FG nup – TF interactions are together sufficient for selectivity.  Our model makes selectivity predictions that will be directly testable in our experimental setup.  We aim to use the model to tune the mimic's parameters to maximize selectivity.