\chapter{Dye-labeling protocols}~\label{appx:dye-labeling}
lookup: what dye chemistry was used for FSFG-A647

The following protocols were used to label proteins with various fluorophores.  Dyes should always be stored in the ultra-low freezer in anhydrous DMSO.  Labeled proteins should be aliquoted within 24 hours of labeling and stored in the ultra-low freezer until just before use.  These protocols were adapted from Thermo-Fisher amine and cysteine labeling protocols with help from Eric Verbeke and Annette Erbse.

\section{Labeling NTF2 with fluorescein-NHS}
\begin{enumerate}
\item Resuspend lyophilized fluorescein-NHS at 100 mg/mL in anhydrous DMSO in the darkroom.  Discard the remaining DMSO aliquot.  Make 100-$\mu$L aliquots of dye solution. Store with desiccant in ultra-low freezer, protected from light.
\item Mix NTF2 in PTB pH 7.0 and 100 mg/mL fluorescein-NHS in DMSO with around 15-fold molar excess dye.  Several other buffers can be used as well (see Thermo protocol).  A typical labeling reaction used 0.5 mL of 16 mg/mL NTF2 and 18.6 $\mu$L dye stock mixed in an Eppendorf with an Eppendorf stir bar.
\item Incubate mixture, stirring, protected from light, at room temperature for one hour.
\item Equilibrate TALON cobalt resin with PTB in a column that can be spun in a centrifuge.  TALON resin has a stated capacity of 5-15 mL protein per mL resin.  Using significantly more than needed can lead to nonspecfic binding of free dye.  A typical reaction required 1.6 mL of the resin slurry.  Equilibrate with at least 10 bed volumes PTB.
\item Cap column and add reaction mixture.  Nutate at 4$^\circ$C for one hour, protected from light.
\item Allow column to drain and wash with approximately 100 bed volumes of PTB, protected from light as best as possible.  Occasionally cap column and resuspend resin to remove free dye from column cap and sides.  Test flow-through using UV light to ensure that no dye is visible by the end of the wash.  Resin should still be bright yellow.
\item Spin column 1000 rpm (convert to rpm, spun in 15 mL conical) for one minute to remove remaining wash buffer.  Immediately cap and elute with 1 mL of 300 mM imidazole in PTB.  Nutate 4$^\circ$C for half an hour and collect elution.  Resin should return to pink.
\item Dialyze elution against PTB to remove imidazole.  No more than 24 hours after labeling, aliquot and freeze labeled NTF2.
\item Run a sample of labeled NTF2 on a native PAGE gel along with a sample of fluorescein-NHS.  Use the Typhoon to compare the concentration of free dye to labeled protein in the protein sample.
\end{enumerate}
\section{Labeling NTF2 with Alexa Fluor 488 - SDP}
\begin{enumerate}
\item Resuspend lyophilized Alexa Fluor 488 at 10 mg/mL in anhydrous DMSO in the darkroom.  Discard the remaining DMSO aliquot.  Make 10-$\mu$L aliquots of dye solution. Store with desiccant in ultra-low freezer, protected from light.
\item Mix NTF2 in 0.1 M sodium bicarbonate buffer and 10 mg/mL fluorescein-NHS in DMSO with around ??-fold molar excess dye.  A typical labeling reaction used 200 $\mu$L of 16 mg/mL NTF2 and 20 $\mu$L dye stock mixed in an Eppendorf with an Eppendorf stir bar.
\item Incubate mixture, stirring, protected from light, at room temperature for one hour.
\item Equilibrate TALON cobalt resin with PTB in a column that can be spun in a centrifuge.  TALON resin has a stated capacity of 5-15 mL protein per mL resin.  Using significantly more than needed can lead to nonspecfic binding of free dye.  Equilibrate with at least 10 bed volumes PTB.
\item Cap column and add reaction mixture.  Nutate at 4$^\circ$C for one hour, protected from light.
\item Allow column to drain and wash with approximately 100 bed volumes of PTB, protected from light as best as possible.  Occasionally cap column and resuspend resin to remove free dye from column cap and sides.  Test flow-through using UV light to ensure that no dye is visible by the end of the wash.  Resin should still be bright yellow.
\item Spin column 500 rpm (convert to rpm, spun in mini centrifuge) for 20 s to remove remaining wash buffer.  Immediately cap and elute with 300 $\mu$L of 500 mM imidazole in PTB.  Nutate 4$^\circ$C for half an hour and collect elution.  Resin should return to pink.
\item Dialyze elution against PTB to remove imidazole.  No more than 24 hours after labeling, aliquot and freeze labeled NTF2.
\item Run a sample of labeled NTF2 on a native PAGE gel along with a sample of fluorescein-NHS.  Use the Typhoon to compare the concentration of free dye to labeled protein in the protein sample.
\end{enumerate}
\section{Labeling NTF2-cys or FSFG-cys with Alexa Fluor 488 - maleimide}
\begin{enumerate}
\item Resuspend lyophilized Alexa Fluor 488 at 10 mg/mL in anhydrous DMSO in the darkroom.  Discard the remaining DMSO aliquot.  Make 10-$\mu$L aliquots of dye solution. Store with desiccant in ultra-low freezer, protected from light.
\item Mix protein in PTB pH 7.0 and 50 $\mu$M TCEP (with no other reducing agent present) and 10 mg/mL dye in DMSO.  Check molar ratios, pg 70 LKM book 5.
\item Incubate mixture, stirring, protected from light, at room temperature for two hours.
\item Equilibrate TALON cobalt resin with PTB in a column that can be spun in a centrifuge.  TALON resin has a stated capacity of 5-15 mL protein per mL resin.  Using significantly more than needed can lead to nonspecfic binding of free dye.  Equilibrate with at least 10 bed volumes PTB.
\item Cap column and add reaction mixture.  Nutate at 4$^\circ$C for one hour, protected from light.
\item Allow column to drain and wash with approximately 100 bed volumes of PTB, protected from light as best as possible.  Occasionally cap column and resuspend resin to remove free dye from column cap and sides.  Test flow-through using UV light to ensure that no dye is visible by the end of the wash.  Resin should still be bright yellow.
\item Spin column 500 rpm (convert to rpm, spun in mini centrifuge) for 20 s to remove remaining wash buffer.  Immediately cap and elute with 300 $\mu$L of 500 mM imidazole in PTB.  Nutate 4$^\circ$C for half an hour and collect elution.  Resin should return to pink.
\item Dialyze elution against PTB to remove imidazole.  No more than 24 hours after labeling, aliquot and freeze labeled NTF2.
\item Run a sample of labeled NTF2 on a native PAGE gel along with a sample of fluorescein-NHS.  Use the Typhoon to compare the concentration of free dye to labeled protein in the protein sample.



0.1 M sodium bicarbonate buffer and 10 mg/mL fluorescein-NHS in DMSO with around ??-fold molar excess dye.  A typical labeling reaction used 200 $\mu$L of 16 mg/mL NTF2 and 20 $\mu$L dye stock mixed in an Eppendorf with an Eppendorf stir bar.
\end{enumerate}
