\chapter{Bis-labeling Nup fragments for incorporation into acrylamide hydrogels}~\label{appx:bis-labeling}

Before FSFG-cys or any other cys-labeled Nup variant can be tethered to an acrylamide hydrogel, it needs to be labeled with bisacrylamide or PEG-DA.  This protocol describes the labeling procedure for either chemical group.  The cysteines must be fully reduced for the labeling to occur.  No BME, TCEP, or other reducing agent can be present in the reaction mixture.  FSFG and other disordered peptides form disulfide bonds within minutes of being removed from reducing agents, so begin the reaction as quickly as possible after reducing.  This protocol was developed with the help of Benjamin Fairbanks.  Following labeling, the extent of labeling can be quantified using an Ellman's reagent assay.

\section{Bis-labeling reaction protocol}
% lots of places in my lab book - book 6 pg 73 provides a starting reference

\begin{enumerate}
\item Begin with a stock of at least 1 mg/mL FSFG with a terminal cysteine in PBS pH 7.8.  Typical reactions use 1 mL of approximately 2 mg/mL FSFG.
\item Equilibrate an equal amount of immobilized TCEP resin slurry in a disposable 5-mL column which can be spun in a conical tube.  Refer to Thermo-Pierce product reference sheet for resin volume and nutation time if necessary.  Equilibrate with 20-30 bed volumes of PBS pH 7.8 using gravity, then spin at 161g for 10 s to remove remaining buffer.
\item Immediately add the FSFG stock and nutate at 4$^\circ$C for 1 hour.
\item Prepare a conical tube with 40 $\mu$L of triethanolamine (TEA, in fume hood)and 50 $\mu$L 2\% bisacrylamide solution.  This will lead to a 10-fold molar excess of bisacrylamide over protein and 300 mM TEA in the final reaction mixture. Place column in conical tube and spin down 1000 rpm for 1 minute.
\item Immediately vortex thoroughly.  Nutate at room temperature for 30 minutes.
\item Dialyze into 25 mM ammonium bicarbonate buffer to remove excess bisacrylamide and prepare for lyophilizing.
\item Perform a BCA to quantify protein concentration. Prepare 100 or 200 $\mu$g aliquots, freeze, and lyophilize.  Store lyophilized protein with a desiccant in ultra-low freezer.
\end{enumerate}

\section{Ellman's reagent assay protocol}
%LKM book 5 pgs 83-111 (got it working properly on 111)

This protocol was taken from Thermo-Fisher's protocol and modified for a microwell plate.  Due to the rapid disulfide bond formation of FSFG, the reactants must be mixed very rapidly once reducing agent is removed.

\begin{enumerate}
\item Equilibrate 80 $\mu$L TCEP resin slurry in each of two disposable 1-mL spin columns which can be spun in an eppendorf centrifuge.  Refer to Thermo-Pierce product reference sheet for resin volume and nutation time if necessary.  Equilibrate with 20-30 bed volumes of PBS pH 7.8 using gravity, then spin at 2300g for 10 s to remove remaining buffer. 
\item Add 80 $\mu$L of bis-labeled FSFG in 25 mM ammonium bicarbonate to one column and a known concentration of unlabeled FSFG cys (as a control) in 25 mM ammonium bicarbonate to the other. Nutate for two hours.
\item While incubation proceeds, prepare the reaction buffer (0.1 M sodium phosphate buffer pH 8.0 with 1 mM EDTA) and the Ellman's reagent solution (ERS, 4 mg/mL Ellman's reagent in reaction buffer).
\item Prepare a 96-well plate with wells containing 1.8 $\mu$M ERS and 29.5 $\mu$M reaction buffer.  Prepare wells for the FSFG-bis and FSFG cys samples but do not add the protein until everything has been prepared.  Prepare a well containing ERS, reaction buffer, and 68 $\mu$L of 25 mM ammonium bicarbonate buffer.  Finally, prepare a well containing ERS, reaction buffer, and 68 $\mu$L of unlabeled FSFG cys of a known concentration that has not be reduced.
\item Spin down both columns of TCEP solution 2300g for 30 s in an eppendorf centrifuge and collect the flow-through.  Very rapidly, add 68 $\mu$L of flow-through to the appropriate wells and mix by pipetting up and down.
\item Incubate plate at room temperature for 15 minutes.
\item Measure the absorbance at 412 nm using a plate reader.
\item Calculate labeling efficiency.  Use the buffer well's absorbance to blank the absorbance of the reduced FSFG cys sample and the labeled FSFG sample.  Divide the blanked absorbance of the labeled sample by that of the reduced FSFG cys sample and subtract the resulting ratio from 1.  The unreduced FSFG sample is not directly used but is a good check on the results.
\end{enumerate}