\chapter{Hydrogel precursor solutions}~\label{appx:precursor-recipes}

This appendix contains details of hydrogel precursor solutions. Lyophilized protein should be resuspended in the buffer component of the precursor solution and incubated at room temperature for at least 20 minutes before adding the other components.  No-Nup control gels simply omit the protein.  Precursor solutions must be degassed for 10 minutes and promptly polymerized.  All solutions containing photoinitiator must be protected from light at all times.  Mix under red light only.  Prepare precursor solutions immediately before use.

\section{PEG hydrogel precursor recipes}
% Sadhana's original workbook on page 73 of LKM book 2
% Equal-volume precursor stock recipes: LKM 4 pg 47
% Nate Crossette tested some hydrogels that were not 10 wt % but these didn't work

All PEG hydrogels were 10 wt \% PEG with and 0.5 thiol-ene ratio.  For more accurate pipetting, stock solutions were designed to be combined in equal volumes (Table~\ref{table:Irgacure-recipe}). 

\begin{table}[b!]
\centering
  \caption[PEG hydrogel precursor stocks.]{PEG hydrogel precursor stocks}
    \label{table:Irgacure}
    \begin{tabular}{p{5cm}p{4cm}p{5cm}}
      Component & Concentration & Buffer \\
\hline
20-kDa PEG-norbornene & 438 $\mu$g/$\mu$L & Water \\
1-kDa PEG-dithiol crosslinker & 45 $\mu$g/$\mu$L & Water \\
8-kDa PEG-dithiol crosslinker & 360 $\mu$g/$\mu$L & Water \\
Irgacure 2959 & 2 mM & Water \\
LAP & 2 mM & Water \\
TCEP & 4 mM & 4x PTB \\
\hline
    \end{tabular}
\end{table}

\begin{table}[b!]
\centering
  \caption[PEG hydrogel recipes.]{Sample precursor soution recipe (10 wt \% PEG, 10 mg/mL nominal Nup concentration)}
    \label{table:Irgacure-recipe}
    \begin{tabular}{p{5cm}p{5cm}}
      Stock & Amount \\
\hline
20-kDa PEG-norbornene & 2.5 $\mu$L \\

1-kDa PEG-dithiol crosslinker & 2.5 $\mu$L \\

Irgacure 2959 & 2.5 $\mu$L \\

TCEP in 4x PTB & 2.5 $\mu$L \\

Lyophilized FSFG-cys & 100 $\mu$g \\
\hline
    \end{tabular}
\end{table}

\section{Acrylamide hydrogel precursor recipes}
The monomer and crosslinker were bought premixed from BioRad (acrylamide/bisacrylamide 30\% 29:1) but potentially could be prepared in different ratios and mixed separately.  Any additional components, such as dextran or photoinhibitor, should be made into a stock with PTB and used instead of the buffer component in the recipe.  Acrylamide hydrogels were mostly crosslinked with LAP photoinitiator, but sometimes with the APS/TEMED chemical crosslinking system.  Results of several APS/TEMED concentrations in a 6\% acrylamide precursor solution are shown in Table~\ref{table:APS}.

\begin{table}[b!]
\centering
  \caption[Acrylamide hydrogel precursor stocks.] {Acrylamide hydrogel precursor stocks}
    \label{table:acrylamide}
    \begin{tabular}{p{5cm}p{4cm}p{5cm}}
      Component & Concentration & Buffer \\
\hline
Acrylamide monomer & 30\% & Premixed \\
Bisacrylamide crosslinker & 1\% & Premixed \\
LAP & 20 mM & Water \\
APS & 10\% w/v & PTB \\
TEMED & 1\% w/v & PTB \\
PTB buffer & 1x & PTB \\
\hline
    \end{tabular}
\end{table}

\begin{table}[b!]
\centering
  \caption[Acrylamide hydrogel recipes.]{Sample precursor soution recipe (6\% final acrylamide concentration)}
    \label{table:acrylamide-recipe}
    \begin{tabular}{p{5cm}p{5cm}}
      Stock & Amount \\
\hline
Premixed acrylamide/bis & 2 $\mu$L \\

LAP & 1 $\mu$L \\

PTB & 7 $\mu$L \\

Lyophilized FSFG-cys & 100 $\mu$g \\
\hline
    \end{tabular}
\end{table}

\begin{table}[t!]
\centering
  \caption[APS/TEMED crosslinking times.]{APS/TEMED chemical crosslinking tests}
    \label{table:APS}
    \begin{tabular}{p{3.5cm}p{4cm}p{2cm}p{5cm}}
      APS concentration & TEMED concentration & Degas time (min) & Results \\
\hline
      1\% & 0.5\% & 0 & Gelled in under 10 s. \\
     0.1\% & 0.5\% & 0 & Gelled in 5 minutes. \\
     0.1\% & 0.1\% & 5 & Gelled in 5 minutes (while degassing). \\
     0.1\% & 0.1\% & 0 & Gelled in 10 minutes. \\
     0.1\% & 0.05\% & 0 & Did not gel. \\
     0.1\% & 0.05\% & 5 & Gelled in 10 minutes. \\
\hline

\hline
    \end{tabular}
\end{table}