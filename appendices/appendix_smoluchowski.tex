\chapter{Derivation of the Smoluchowski equation}~\label{appx:smoluchowski}
Here, I present a derivation of the Smoluchowski equation starting from the
Langevin equation~\cite{zwanzig_nonequilibrium_01}. The Smoluchowski equation describes the dynamics of the $N$
particle noise ensemble density, $\rho^{N}(\bm{r}^N)$ where $\bm{r}^N = [
\bm{r}_1, \bm{r}_2, \dots \bm{r}_N]$. The Langevin equation is
%
\begin{equation}
  \label{eqn:full_langevin}
   \dot{\bm{r}}_i = \zeta ^{-1}
     \left[ -\bm{\nabla}_i U( \bm{r}^N )
       + \bm{\xi}_i(t) \right],
\end{equation}
%
where $i$ is the particle label, $U$ the potential energy, and $\bm{\xi}_i$ the
noise. Assuming the friction is isotropic, homogeneous, and does not mediate
hydrodynamic interactions between particles, the noise has properties
% Noise
\begin{gather}
  \langle \xi_i( t ) \rangle = 0, \\
  \langle \xi^{\alpha}_i( t ) \xi^{\beta}_j( t') \rangle 
  = 2 \zeta \delta_{ij} \delta_{\alpha \beta} \delta(t-t'),
\end{gather}
%
where $\alpha, \beta$ refer to dimensions and $i,j$ to particle label.  Consider
the $N$ particle phase space density $f(\bm{r}^N,t)$, which describes the
probability density of finding particles between $\bm{r}^N$ and $\bm{r}^N+\delta
\bm{r}^N$.  Note that this does not depend on momentum  because we are in the
overdamped limit.  In other words, we are averaging over the momentum degrees of
freedom because we assume their dynamics equilibrate on time scales we are not
interested in~\cite{archer_dynamical_04}.

The probability density is conserved.  Therefore, it obeys a continuity equation
%
\begin{equation}
  \label{eqn:continuity_phase}
  \frac{d f (\bm{r}^N,t)}{dt} 
  = \diff{f (\bm{r}^N,t)}{t} 
  + \bm{\nabla}_i \cdot \left[ \bm{r}^N_i f(\bm{r}^N,t) \right] = 0,
\end{equation}
%
where the Einstein summing convention is implied.  In the absence of noise, this
is the Liouville equation~\cite{zwanzig_nonequilibrium_01}.  Including  noise
leads to the Fokker-Planck equation.  Let $\bm{v}_i = -\nabla_i U(\bm{r}^N) $,
and plug \eqnref{eqn:full_langevin} into \eqnref{eqn:continuity_phase} to get
%
\begin{equation}
   \diff{f(\bm{r}^N,t) } {t} 
   + \bm{\nabla}_i \cdot \zeta ^{-1} \left[ f(\bm{r}^N, t) \left(
     \bm{v}_i(\bm{r}^N) +
     \bm{\xi}_i(t) \right) \right] = 0.
\end{equation}
%
We can separate the noise term to get
%
\begin{equation}
  \label{eqn:homo_inhomo_smol}
   \diff{f}{t} = - \mathcal{L} f 
   - \bm{\nabla}_i \cdot \zeta^{-1} \bm{\xi}_i(t) f,
\end{equation}
%
where $\mathcal{L} = \zeta^{-1}_i \bm{\nabla}_i \bm{v}_i $.  Integrating
with respect to time gives
%
\begin{equation}
  \label{eqn:fp_propagator}
  f(t) = e^{-\mathcal{L} t} f(0)
  - \zeta^{-1} \int_0^t \dif s \, e^{-\mathcal{L}(t-s)} 
  \bm{\nabla}_i \cdot f(s) \bm{\xi}_i(s).
\end{equation}
%
We plug \eqnref{eqn:fp_propagator} back into \eqnref{eqn:homo_inhomo_smol} and
take the noise average $\langle \ldots \rangle$, which gives
%
\begin{align}
  \diff{\langle f(t) \rangle}{t} =&
  -L \langle f(t) \rangle
  + \bm{\nabla_i} \cdot \zeta^{-1} 
  \langle \bm{\xi}_i \rangle e^{-\mathcal{L} t} f(0) \\
  -& \bm{\nabla}_i \cdot \zeta^{-2} 
  \int_0^t \dif s \, e^{-\mathcal{L}(t-s)} 
  \bm{\nabla}_j \cdot 
  \left \langle \bm{\xi}_i(t) f(s) \bm{\xi}_j(s) \right \rangle.
\end{align}
%
Before we can use the properties of the noise to deal with the integral, we
consider the $\langle \xi_i(t) \xi_j(s) f(s) \rangle$ term. The term $f(s)$
depends on the noise terms at earlier times, and the noise terms are
delta-function correlated in time.  Thus, there cannot be correlations between
the noise at $s$ and $t$ and the phase space density because $f(s)$ only depends
on noise at earlier times.  Therefore, $\langle \xi_i(t) \xi_j(s) f(s) \rangle =
\langle \xi_i(t) \xi_j(s) \rangle \langle f(s) \rangle = \langle \xi_i(t)
\xi_j(s) \rangle \rho^{N}(s)$, where $\rho^{N}(\bm{r}^N,s) = \langle f(s)
\rangle$ is the noise averaged phase space probability density.  Making this
substitution and using properties of the noise we get the Smoluchowski equation,
%
\begin{equation}
  \diff{\rho^{(N)}(\bm{r}^N,t) }{t} =
  \bm{\nabla}_i \cdot
  \left[ \zeta^{-1} \bm{\nabla}_i U(\bm{r}_N) \rho^{(N)}(\bm{r}^N, t) \right] 
   + \bm{\nabla}_i \cdot D \bm{\nabla}_i  
   \rho^{(N)}(\bm{r}^N, t),
\end{equation}
%
In the absence of fields, this is the diffusion equation. When the potential
energy arises from an external field $U(\bm{r}^N)=\sum_i \phi(r_i)$, we recover
the drift-diffusion equation. If more complicated friction tensors are involved,
the equation becomes~\cite{archer_dynamical_04}
%
\begin{equation}
  \diff{\rho^{(N)}(\bm{r}^N,t) }{t} =
  \bm{\nabla}_i \cdot
  \left[ \bm{\zeta}_{ij}^{-1} \bm{\nabla}_j 
    U(\bm{r}_N) \rho^{(N)}(\bm{r}^N, t) \right] 
   + \bm{\nabla}_i \cdot \bm{D}_{ij} \bm{\nabla}_j
   \rho^{(N)}(\bm{r}^N, t).
\end{equation}
%
