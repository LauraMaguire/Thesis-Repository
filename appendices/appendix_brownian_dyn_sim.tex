\chapter{Numerics for Brownian dynamics simulations}~\label{appx:brownian_dyn_sim}
Here, I connect the Langevin equation and Brownian dynamics
simulations~\cite{grassia_computer_95}.  The Brownian equation of motion is
%
\begin{equation}
  \dot{x} = - \zeta ^{-1} \left[  f(x,t) + \xi(t) \right],
\end{equation}
%
where the force $ f(x,t) = -\frac{dU}{dx}  $ depends on time and the noise has
the properties
% Noise
\begin{gather}
  \langle \xi( t ) \rangle = 0,\\
  \label{eqn:noise_delta}
  \langle \xi( t ) \xi( t') \rangle 
  = 2 \zeta k_B T \delta(t-t').
\end{gather}
%
Integrating from $t$ to $t+\delta t$ gives
%
\begin{equation}
  x(t+\delta t) = x(t) - \zeta ^{-1} 
  \int_t^{t+\delta t} \dif t' f(x,t') + \xi(t').
\end{equation}
%
To numerically evaluate this, we approximate the delta function in the
noise term variance~\eqnrefp{eqn:noise_delta} as
%
\begin{equation}
  \delta(t-t') \approx \delta(t,t') = 
  \begin{cases}
    \frac{1}{\delta t}, &\text{if $t$ and $t'$ are in the same
  time step $\delta t$} \\
    0, & \text{otherwise}
  \end{cases}.
\end{equation}
%
Thus, upon integration we get
%
\begin{equation}
  x(t+\delta t) \approx x(t) - \zeta ^{-1} 
  \delta t \left[ f(t) + c(t) \sqrt{\frac{2 \zeta k_B T}{\delta t}}  \right].
\end{equation}
%
where $c(t)$ is a random number selected from a distribution with average zero
and variance 1.  Typically, we select a number from the uniform distribution
$r \in [-1/2, 1/2] $ which has a variance of $1/12$.  Thus, the position at time
$t_{n+1}$ is
%
\begin{equation}
   x_{n+1} \approx x_{n} - \zeta ^{-1} 
  \delta t f_n + \sqrt{24 D \delta t} r_n,
\end{equation}
%
where we have used the Einstein relation $D = \frac{k_B T}{\zeta}$. At each time
step, $f_n$ is calculated, and $r_n$ is selected.  In the absence of forces, the
MSD is
%
\begin{equation}
   \langle x_{N} ^2  \rangle = 24 D \delta t \langle r^2 \rangle N = 2Dt,
\end{equation}
%
after $N = t / \delta t$ steps.  A step is randomly selected from a uniform
distribution  $\Delta x \in [ -\sqrt{6 D \delta t}, \sqrt{6 D \delta t} ] $ in a
Brownian dynamics simulation.  Note that one can select a random
number from a Gaussian distribution instead of a uniform distribution.  In that
case, a random step length is selected from a Gaussian distribution with average
$\mu = 0$ and variance $\sigma = \sqrt{2 D \delta t }$. 
