\chapter{Derivation of DDFT}~\label{appx:ddft}
This appendix describes how to go from the Smoluchowski equation to a closed
form for the one-particle density using dynamic density functional
theory~\cite{archer_dynamical_04}.  The Smoluchowski equation describes the
evolution of the $N$ particle density $\rho^{(N)}(\bm{r}^N,t)$,  the probability
of finding the particles near $\bm{r}^N = [ \bm{r}_1, \bm{r}_2, \ldots
\bm{r}_N]$ in phase space, is
\begin{equation}
  \diff{\rho^{(N)}(\bm{r}^N,t) }{t} =
  \bm{\nabla}_i
  \left[ \zeta_{ij}^{-1} \bm{\nabla}_j U(\bm{r}^N,t) 
    \rho^{(N)}(\bm{r}^N, t) \right] 
   + \bm{\nabla}_i D_{ij} \bm{\nabla}_j
   \rho^{(N)}(\bm{r}^N, t),
\end{equation}
%
where $U$ is the potential. In general, 
%
\begin{equation}
  U( \bm{r}^N, t ) = \sum_{i=1}^N V_{\text{ext}}(\bm{r}_i,t)
  + \frac{1}{2} \sum_{j \neq i} \sum_{i=1}^N v_2( \bm{r}_i, \bm{r}_j )
  + \frac{1}{6} \sum_{k \neq j \neq 1} \sum_{j \neq i} 
  \sum_{i=1}^N v_3( \bm{r}_i, \bm{r}_j, \bm{r}_k )
  + \ldots
\end{equation}
%
where $V_{ext}$ is the potential from external fields and $v_n$ is the $n$-body
interaction potential.  We are deriving an equation for the
one-body density
%
\begin{equation}
  \rho^{(1)}(\bm{r}) = N \int \dif \bm{r}_2 \ldots \int \dif \bm{r}_N 
  \rho^{(N)}( \bm{r}^N ),
\end{equation}
%
which describes the probability of finding a particle at position $\bm{r}$
irrespective of the other particles' locations.  In general, the $n$-body
particle density is~\cite{hansen_theory_06}
%
\begin{equation}
  \rho^{(n)}(\bm{r}^n) = \frac{N!}{(N-n)!} \int \dif \bm{r}_{n+1} 
  \int \dif \bm{r}_N 
  \rho^{N}( \bm{r}^N , t ).
\end{equation}
%
We can integrate the Smoluchowski equation over coordinates $\bm{r}_2, \ldots
\bm{r}_N$ to get 
%
\begin{align}
  \label{eqn:pde_open}
  \diff{\rho^{(1)}(\bm{r}_1,t) }{t} =&
  \bm{\nabla}_1 D \bm{\nabla}_1  
   \rho^{(1)}(\bm{r}_1, s)
  + \zeta^{-1} \bm{\nabla}_1 
  \left( \rho^{(1)}(t) \bm{\nabla}_1 V_{\text{ext}} \right) \\
  &+ \bm{\nabla}_1 D \int \dif \bm{r}_2 \rho^{(2)}(\bm{r}_1,\bm{r}_2) 
  \bm{\nabla}_1 v_2 (\bm{r}_1, \bm{r}_2 ) \\
  &+ \bm{\nabla}_1 D \int \dif \bm{r}_2 \int \dif \bm{r}_3 
  \rho^{(3)}(\bm{r}_1,\bm{r}_2,\bm{r}_3) \bm{\nabla}_1 v_3
  (\bm{r}_1, \bm{r}_2, \bm{r}_3) + \ldots
\end{align}
%
This is not a closed equation because the evolution of $\rho^{(1)}$ depends on
the higher order densities $\rho^{(2)}, \ldots, \rho^{(n)}$. Evans found a way
to express the two-body density in terms of the direct correlation function
$-k_B T \rho^{(1)}(\bm{r}_1) \bm{\nabla}_1 c^{(1)}(\bm{r}_1) = \int d \bm{r}_2
\rho^{(2)}(\bm{r}_1,\bm{r}_2) \bm{\nabla}_1 v_2 (\bm{r}_1, \bm{r}_2 )$ for a
liquid in equilibrium~\cite{evans_nature_79}.  This can be generalized for
higher order terms~\cite{archer_dynamical_04} to give
%
\begin{equation}
  \label{eqn:dir_corr_sum}
  -k_B T \rho^{(1)}(\bm{r}_1) \bm{\nabla}_1 c^{(1)}(\bm{r}_1) = 
  \sum_n \int \dif \bm{r}_2 \ldots 
  \int \dif \bm{r}_n \rho^{(n)} \bm{\nabla}_1 v_n (\bm{r}^n).
\end{equation}
%
Plugging this into \eqnref{eqn:pde_open} gives a closed form for our equations
if $c^{(1)}$ is known and not dependent on
higher order densities.  The direct
correlation function in equilibrium can be expressed as a functional derivative
of the excess free energy, \textit{i.e.}, free energy contribution from
interactions, with respect to the single body free
energy~\cite{evans_nature_79, hansen_theory_06, marconi_dynamic_99,
  marconi_dynamic_00}
%
\begin{equation}
  \label{eqn:dir_corr_functional}
  c^{(1)}(\bm{r}) = -\beta \frac{\delta \mathcal{F}^{\tx{ex}} 
      \left[ \rho^{(1)}(\bm{r}) \right] }
  {\delta \rho^{(1)} (\bm{r} ) }.
\end{equation}
%
This assumes that \eqn~\ref{eqn:dir_corr_sum} and~\ref{eqn:dir_corr_functional}
are valid for a fluid out of equilibrium~\cite{archer_dynamical_04}.  This gives
the DDFT equation of motion~\cite{evans_nature_79,
  marconi_dynamic_99, marconi_dynamic_00, archer_dynamical_04}
%
\begin{equation}
  \frac{\partial \rho(\bm{r},t)}{\partial t} = 
  \bm{\nabla} \cdot \left[ \zeta^{-1} \rho(\bm{r},t) \bm{\nabla} 
    \frac{\delta \mathcal{F} \left[ \rho (\bm{r},t) \right] }
  {\delta \rho (\bm{r}, t ) } \right],
\end{equation}
%
where we have dropped the superscript $(1)$ from the one-body density.  The free
energy is separable into contributions from the ideal gas entropy,
excess/interactions, and external fields~\cite{hansen_theory_06}
%
\begin{equation}
  \mathcal{F} = \mathcal{F}^{\tx{id}} + \mathcal{F}^{\tx{ex}}
  +\mathcal{F}^{\tx{ext}}.
\end{equation}
%
The ideal gas and external contributions are exact~\cite{hansen_theory_06,
  archer_dynamical_04}:
%
\begin{equation}
  \mathcal{F}^{\tx{id}} = k_B T \int \dif \bm{r}^d \rho(\bm{r} )
  \left[\ln{\Lambda^d \rho(\bm{r} )} - 1\right],
\end{equation}
%
where $\Lambda$ is the thermal de Broglie wavelength and
%
\begin{equation}
  \mathcal{F}^{\tx{ext}} = \sum^N_{i=1} V_{\tx{ext}}(\bm{r}_i, t).
\end{equation}
%
The excess free energy is not necessarily known and may be
approximated~\cite{hansen_theory_06}.  DDFT, although built on the stochastic
Langevin equation, is deterministic~\cite{archer_dynamical_04a} because we have
taken the noise ensemble average.
