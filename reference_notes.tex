Tu and Musser, 2011, Single-molecule review

They have a table with interaction times for various TFs and cargo.
Short discussion of diffusion constant within the pore.
Long discussion of analytic models (Rout, Zilman, etc.) including jamming, passive transport, facilitated release, and partition coefficients.
Some experimental results for the effect of jamming by Kaps.

Kinetics of transport through the nuclear pore complex
Ulrich Kubitscheck, Jan-Peter Siebrasse, 2017

Introduction says that RNA export uses different transporters and energy sources than Kaps and Ran.
Article focuses on single-molecule studies of impBeta and NTF2
Can get protein into cells by microinjections, digitonin permeabilization, or overexpression of recominant gene, or autocatalytic labeling like SNAP.
New NTF2 flux measurement? ``Recombinant NTF2 alone binds to and translocates across the NPC with very high bulk transport rates of 1500–2500 NTF2-dimers per second and NPC in vitro''
Reference to Herrmann study of NTF2 Brownian motion in pore (estimated diffusion constant in pore is 17 um^2/s)
Lots about Kap, esp. two-population (strong and weak binding) inside the pore.
Explanation of mRNA export (``The principal mRNA export receptor in yeast is Mex67, its metazoan counterpart is called TAP or NXF1 (Nuclear export factor 1) and both proteins hetero-dimerize via their NTF2-like domain with a small interaction partner: Mex67 with Mtr2 and TAP with NXT1/p15.'')