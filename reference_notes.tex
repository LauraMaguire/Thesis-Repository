\section{Tu and Musser, 2011, Single-molecule review}

They have a table with interaction times for various TFs and cargo.
Short discussion of diffusion constant within the pore.
Long discussion of analytic models (Rout, Zilman, etc.) including jamming, passive transport, facilitated release, and partition coefficients.
Some experimental results for the effect of jamming by Kaps.

\section{Kinetics of transport through the nuclear pore complex, Ulrich Kubitscheck, Jan-Peter Siebrasse, 2017}

Review
Introduction says that RNA export uses different transporters and energy sources than Kaps and Ran.
Article focuses on single-molecule studies of impBeta and NTF2
Can get protein into cells by microinjections, digitonin permeabilization, or overexpression of recominant gene, or autocatalytic labeling like SNAP.
New NTF2 flux measurement? ``Recombinant NTF2 alone binds to and translocates across the NPC with very high bulk transport rates of 1500–2500 NTF2-dimers per second and NPC in vitro''
Reference to Herrmann study of NTF2 Brownian motion in pore (estimated diffusion constant in pore is 17 um^2/s)
Lots about Kap, esp. two-population (strong and weak binding) inside the pore.
Explanation of mRNA export (``The principal mRNA export receptor in yeast is Mex67, its metazoan counterpart is called TAP or NXF1 (Nuclear export factor 1) and both proteins hetero-dimerize via their NTF2-like domain with a small interaction partner: Mex67 with Mtr2 and TAP with NXT1/p15.'')

\section{Protein Transport by the Nuclear Pore Complex: Simple Biophysics of a Complex Biomachine, Tijana Jovanovic-Talisman and Anton Zilman, 2017}

Review
Discussion of Ran cycle
Binding site valencies for NTF2 and Kaps
Discussion of Nup cohesiveness
Discussion of consensus features of computational modeling
Section on NPC mimics

\section{Floppy but not sloppy: Interaction mechanism of FG-nucleoporins and nuclear transport receptors, Iker Valle Aramburu, Edward A. Lemke, 2017}

Review
Estimate of FG concentration within pore
Conserved properties of FG nups
Figure: structures of NTRs bound to cargo
Discussion of NTR-Nup binding sites and CRM binding mechanisms
Ultrafast kinetics

\section{Ribbeck and Gorlich 2001, kinetic analysis}

We use this as a source of NTF2 flux data in the paper.  Also contains estimate of total mass flow and a table for various protein fluxes.  Estimates of number and volume of NPCs in HeLa nuclei.

Good for flux section in experimental observations

\section{Timney 2016, passive barrier, ``Simple biophysics...''}

Talks about rigid vs soft barrier, previous estimates of firm size cutoffs.
Permeability coefficient is worth understanding better - it's a flux measurement but slightly dressed up.

``A gradual and approximately cubic dependence on molecular mass''

- not finished summarizing here; I was looking for actively-transported mass flow estimates and there aren't any.  Very important for passive barrier section though.

Siebrasse 02 and Kiskin 03 are the other references we use for NTF2 flux in the modeling paper.

From Kapinos 2017: ``Approximately 1,000 selective translocation events ensue per NPC per second in both direc- tions (Ribbeck et al., 1998) [is this a mis-citation intended for Ribbeck 01?], where 100 Kapβ1 molecules are estimated to occupy the pore at steady state (Paradise et al., 2007; Tokunaga et al., 2008; Lowe et al., 2015). ``

Kapinos 17 will be good for Kap-centric sections of experimental observations.

http://emboj.embopress.org/content/17/22/6587 Ribbeck and Gorlich 1998 NTF2 and Ran cycle

I checked the Lowe 15 paper - describes in terms of fluorescence counts; estimates 70 or fewer Kaps in the pore at once.  Checked the Tokunaga paper and it checks out.  

\section{Diffusion constants in vitro}

From: Significant proportions of nuclear transport proteins with reduced intracellular mobilities resolved by fluorescence correlation spectroscopy. (Paradise 07)
J Mol Biol. Author manuscript; available in PMC 2008 Jan 5.
Published in final edited form as:
J Mol Biol. 2007 Jan 5; 365(1): 50–65.
Published online 2006 Oct 4. doi: 10.1016/j.jmb.2006.09.08
Copyright/License ►Request permission to reuse
Table I
FCS analysis of nuclear transport proteins in vitro

MW (kD)	τ (μs)	D (μm2/s)
Importin β	97	460	51
Importin α	58	349	68
Ran	23	227	104
NTF2	15	218	108

This paper also has estimates of various TF concentration in nucleus and cytoplasm.

``When the FCS observation volume was positioned over the nuclear envelope significant numbers of immobile molecules were detected for each nuclear transport protein (~104 importin β molecules/pore, ~48 importin α molecules/pore, ~43 Ran molecules/pore and ~6 NTF2 molecules/pore) (Figure 7). ``
















