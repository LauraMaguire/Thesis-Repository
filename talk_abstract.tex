Interactions between particles and their environment can greatly alter the
dynamics of biological systems. In crowded media, like the cell, interactions
with obstacles can introduce anomalous subdiffusion. Active matter
systems---\textit{e.g.}, bacterial swarms---are nonequilibrium fluids where
interparticle interactions and activity cause collective motion and dynamical
phases. In this talk, I will discuss my advances in the fields of crowded media
and active matter. For crowded media, I study the effects of soft obstacles and
bound mobility on tracer diffusion using a lattice Monte Carlo model. I
characterize how bound motion can minimize the effects of hindered anomalous
diffusion, which has implications for protein movement and interactions in
cells.  I extend our analysis of binding and bound motion to study transport
across biofilters like the nuclear pore complex (NPC). Using a minimal model, I
make predictions on the selectivity of the NPC in terms of physical parameters.
Finally, I study the temporal evolution of an active matter system using
dynamical density functional theory.  I map out a dynamical phase diagram for
self-propelled needles and discuss the connection between a banding instability
and microscopic interactions.


Interactions between particles and their environment can alter the dynamics of
biological systems. In crowded media, like the cell, interactions with obstacles
can introduce anomalous subdiffusion. Active matter systems---\textit{e.g.},
bacterial swarms---are nonequilibrium fluids where interparticle
interactions and activity cause collective motion and dynamical phases. In
this talk, I will discuss my advances in the fields of crowded media and
active matter. For crowded media, I study the effects of soft obstacles and
bound mobility on tracer diffusion and transport across biofilters like the
nuclear pore complex. For active matter, I study the dynamical phases of a
self-propelled needle system using dynamical density functional theory.
