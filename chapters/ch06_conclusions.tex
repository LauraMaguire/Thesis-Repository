\chapter{Conclusions and future directions}~\label{ch06_conclusions}

Living systems contain many examples of selective filters which control the transport of all manner of macromolecules.  In this work, we studied biofilters which allow rapid transport of highly specific targets, counterintuitively relying on binding interactions to promote rather than hinder flux.  These filters often contain similar features, such as transient, multivalent binding and binding to flexible, dynamic tethers.  In order to understand the presence of these features, we approached the problem through both modeling and experiment.  Bound-state diffusion proved to be a key parameter for selectivity.  While the theory and experimental setup were based on nuclear transport, they were not designed to exactly reproduce the nuclear pore, but to be applicable to a wider variety of biofilters.

We created a minimal model of the nuclear pore, containing very little beyond the diffusion of transport factors and their binding to FG Nups.  Interestingly, even this highly simplified model was able to reproduce the selectivity properties of the nuclear pore, predicting that binding could increase the flux of a protein through this material up to 300-fold over the flux of an identical inert protein.  The simplicity of our model made the key parameter clear: we could not reproduce the selectivity of the nuclear pore without allowing bound-state diffusion, that is, assuming that bound Nup - transport factor complexes were mobile within the pore.  Plausible mechanisms of bound diffusion in the nuclea pore include tethered diffusion arising from the disordered, flexible nature of the Nups, and inter-chain hopping of the transport factor due to its binding multivalency.  We modeled both of these mechanisms and found that they allow for significant bound mobility and therefore high selectivity.  These mechanisms may also be at work in other highly-specific, high-throughput biofilters.

After developing the minimal model of selectivity, we began developing a synthetic biofilter inspired by the nuclear pore with which to experimentally probe the effect of bound-state diffusion  on selectivity.  In order to capture the key features of selectivity, we designed hydrogels containing peptide tethers derived from FG Nups.  The transport factor NTF2 served as a test protein whose diffusion we could compare with a similarly-size but nonbinding counterpart.  Fluorescence recovery after photobleaching (FRAP) was used to determine the effective diffusion constants of both NTF2 and an inert protein, from which the bound diffusion constant was calculated. We measured a non-zero bound diffusion constant that is consistent with the predictions of our model, and tested the effect of varying Nup length on bound-state diffusion.

Additionally, we probed the aggregation behavior of an FG-Nup-derived peptide in several crowded conditions.  We used a fluorescent aggregation assay as well as NMR and fluorimetry to investigate differences in the aggregated state between crowding conditions, and found that even inert crowders which are widely used interchangeably show differences in the local chemical environments of the peptide.

Although a true hydrogel-based selective biofilter proved challenging to design, our nuclear-pore-inspired material can be used to measure the bound-state diffusion of proteins.  This parameter is likely important to a variety of problems involving rapid transport of highly-selective proteins.  The model and biomaterials developed here could be used to investigate these systems more generally in the future.  This work suggests that bound-state diffusion, particularly when resulting from transient, multivalent binding, may help explain a number of unusual biological filters.

\section{Future directions}

It would be interesting to apply our model of bound-state diffusion to other selective biofilters, particularly those, like the ones described in Chapter~\ref{ch01_introduction}, which share similarities with the nuclear pore.  One question which we are well-positioned to investigate using both our theoretical model and experimental setup is particle diffusion in mucus \cite{witten17,newby18}.  As described in Sec.~\ref{sec:mucus}, the mucus barrier in the lungs is an obstacle to drug delivery, because nanoparticles tend to get trapped in this barrier, which is constantly being cleared and replaced.  Our bound-state diffusion model of flux through a barrier could be used to describe the situation of nanoparticles which interact with the mucus layer, particularly if we modify it to include a mucus flux representing the clearance of mucus.  It is currently not clear whether drug delivery is better accomplished with particles designed to bind to mucins or to avoid interactions \cite{schneider17,lai11}.  Our model broadly predicts that the flux of nanoparticles through the mucus layer will be higher with binding but only if the binding mechanism permits high mobility of the bound complex.  A first step towards testing our model is measuring the diffusion of nanoparticles in mucus, an experiment that could be done using the flow chambers developed in Chapter~\ref{ch03}.  It would likely be possible to use both the model and experimental systems with only minor modifications to investigate nanoparticle diffusion and drug delivery through mucus barriers.

Another system with striking similarities to our bound-diffusion model of selectivity is the speed-stability paradox of transcription factor and DNA damage repair proteins that target specific sites on DNA (Sec.~\ref{sec:DNA-diffusion}).  In the case of the DNA damage repair protein poly(ADP-ribose) polymerase 1 (PARP1) in particular, PARP1 binds even intact chromatin with micromolar affinity, but it moves to the sites of DNA damage with a diffusion constant only an order of magnitude less than its diffusion in water \cite{rudolph18,mahadevan18}.  If there is truly only negligible free PARP1 in the nucleus, and it is almost entirely bound to DNA, then all diffusion to sites of damage must be bound diffusion.  Given the high diffusion constant measured, we might then expect to see evidence of the mechanisms which we think give rise to bound diffusion in the nuclear pore complex: multivalent binding and tethered diffusion.  

Indeed, PARP1 contains four DNA binding domains and has been shown to possess a `'hopping'' mechanism to move between stands while bound \cite{rudolph18}.  Furthermore, recent unpublished data from Johannes Rudolph and the Luger group suggest that PARP1 may dimerize while bound to intact DNA searching for damage.  Previous work has also indicated that PARP1 may dimerize, though the context and extent of dimerization is unclear \cite{pion05,altmeyer09}.  At first glance, our model predicts that dimerization could increase the bound diffusion constant, provided it increases the rate of hopping between strands.  Multivalent binding appears to be important to PARP1 diffusion in several ways.  Tethered diffusion is also a plausible mechanism in this case, as there is evidence that chromatin is dynamic and could act as a flexible tether \cite{nozaki17,maeshima16}.  Naive calculations which use only our  tethered diffusion model and do not allow hopping are consistent with the calculated off-rate from intact DNA (Sec~\ref{sec:parp1}).  

Finally, it is interesting that PARP1 binding to DNA is another example of a highly multivalent system, with several binding sites on each PARP1 and many on each DNA strand, for which in vivo affinity measurements are very difficult.  Even affinity measurements to intact DNA as opposed to damaged are difficult to carry out in vitro, as most DNA in vitro is damaged somehow.  This situation reminds me of the similar historical difficulty in measuring Nup - transport factor affinities, and the way that the field's understanding of the mechanisms of nuclear transport evolved as those measured values improved.   If the problem of binding affinity in the nuclear pore is any indication, I would not be surprised to find that future calculations of PARP1's affinity for intact DNA show weaker binding than is currently accepted. 

The similarities between the nuclear pore and PARP1 diffusion are fascinating, as is the problem of diffusion through a mucus barrier, and our model of selectivity is well-positioned to offer insight into all of these.  While ideas of bound-state diffusion are already present in all of these fields, I think our model has particular value in clarifying why bound-state diffusion can lead to unusual selective filters and in suggesting practical mechanisms with which to engineer systems with bound diffusion.  There are a surprising number of examples of selective biofilters which using binding to increase flux, and our model provides new insight into how they might accomplish this remarkable task.

%%%%%
%One way to apply the model and some of the experiment is to look at particle diffusion in mucus.  As described in the introduction, it's difficult to get drugs through the mucus barrier of the lungs, both because they get trapped there and because the mucus is contantly being pushed away from the cells and cleared.  We might be able to look at this both with the model and the experiment.  If we added a flux `in the wrong direction' to the model, it could be applied better to mucus layers.  We could see how much bound diffusion we need to overcome the flux, and then experimentally try to create nanoparticles which have ways to diffuse while bound to mucus, or in mucus.  This would require maybe using the flow chambers that I developed.  Particularly we can use the x-shaped chamber and put a hydrogel window down the middle.  Then mucus can stay in one arm while various solutions are flowed past the other arm.  We could measure the bound diffusion into the mucus using some of my analysis, and/or FRAP, and/or brownian motion of nanoparticles.  This would be interesting because it would be an attempt to apply bound diffusion where there currently isn't any, and improve selective drug delivery through a mucus barrier.

%One exciting possible direction is the model as applied to PARP1 and other transcription factors.  In the case of PARP1 in particular, the more we learn about it, the more it has in common with our model.  First of all, it clearly has multivalent binding that lets it hop from one strand of DNA to another, as shown by Johannes.  The DNA/chromatin is a flexible linker like the FG Nups, and there is a fair amount of evidence that it is pretty dynamic.  So already we've got both the tethered diffusion and hopping mechanisms in play.  On top of that, the consensus of the field is that PARP1 spends approximately all of its time in the nucleus bound to DNA, so bound-state diffusion must be the only way for it to find DNA damage.  It has a very high bound diffusion constant in that case, so it is necessarily really interesting to study with our model.  Very recently, Johannes ran some experiments that suggest PARP1 might dimerize but only when bound to intact DNA.  This is exactly what we expect from our model, which says that increasing the hopping rate, presumably by increasing the binding valency, should increase the bound diffusion constant.  And then when it arrives at DNA damage, if it stops being a dimer, that could help it stop moving around and stick where it is supposed to be.  There are also echoes of the nuclear pore field in the questions surrounding binding affinity of PARP1 and transcription factors to DNA.  Questions of extreme multivalency arise again, as it's not clear even how to count the number of distinct possible binding sites along a strand of DNA, but there is clearly a really lot of it in the nucleus.  And PARP1 has up to four DNA binding sites, but some might work in tandem, and it's hard to isolate the effect of each one.  And it's very difficult to get affinity measurements for intact DNA, because almost all the DNA we can make in the lab has some form of damage.  So if the current paradox is that binding affinity is very tight, but PARP1 moves quickly, that sounds a lot like the paradox in the nuclear pore field around binding, which was more or less resolved by deciding that binding was in fact much weaker than previously thought.  I don't know if that's likely to happen in the PARP1 field, but it's an interesting parallel.  Overall, I think that it makes a lot of sense to keep trying to apply our model of bound-state diffusion to PARP1 and maybe transcription factors in general.  The more we dig into it, the more fascinating similarities I see.

%In conclusion, I really do think this model is interesting.  I think a lot of people are working with parts of it, but not many people are looking at it quite like we are through the lens of bound-state diffusion.  The Olsen group is (but that's kinda sketchy) and Katharina Ribbeck, maybe Anton Zilman.  That's not many people in the nuclear pore field.  I don't know as much about mucus-diffusion and transcription-factor people, but my quick look over review articles doesn't strike me as though the idea of bound-state diffusion through tethers and hopping is all that common, at least not in those words.  I could be wrong.  But I hope that somehow it becomes common.


%%%%%%
%I'm stuck again so it's time to just type and see what happens.  These are supposed to be big-picture things that I'm excited about applying the model to.  Really it's just the model.  The experiments didn't work.  Maybe mucus.  That's pretty much the one thing on the horizon that the experiments could help with, or the flow chambers anyway.  We could make flow chambers containing mucus and see how fast things get into it.  Yay.
%
%This is supposed to be stuff that I'm actually excited about.  I actually am probably the most interested in the PARP1 stuff, although I'm probably wrong to be interested. I tend to be wrong about what's interesting.  It just feels like it has all these similarities to the nuclear pore, and we keep finding more the more we look at it.  Like, there were already tethers and multivalency, plus the monkey-bar model as a hopping mechanism that Johannes worked out in detail. And then when we asked if chromatin actually moves like a tether, there were immediately all these papers saying it probably does?  And they are old papers, too, though I don't really know what that means from an is-this-interesting standpoint.  And if we're looking for a way for it to move faster when it's not at its active site, Johannes' dimer evidence is pretty much exactly what we'd think to find.  Plus, and this has always been a bit of a stretch, but even the thing about how everyone thinks it's like nanomolar binding but can't figure out how it can still move around.  That feels a lot like the nuclear pore before people started working out that it was actually really weak binding.  Because mostly we care about how it interacts with intact DNA, and, as Johannes points out often, that's really hard to study in the lab.  So it kind of makes sense to me that most people might be wrong in their affinity measurements, although there's not really a way to say that officially.  So it just seems to me like we keep encountering more information that makes total sense if our model is right, and also that the field as a whole is confused.  So why wouldn't we keep trying to put a paper together?
%
%Okay, can I work myself up to get excited about the mucus stuff too?  I've been reading some stuff about it but I know Loren has learned a lot more for that grant proposal.  Maybe that's why he's enthusiastic about it.  Realistically I'm not going to read much more about it at the moment beyond what I looked up for the intro.  So okay.  Drugs have to get through the mucus layer to be useful.  That's difficult.  Plus mucus is actually flowing the wrong way, to make it even more difficult.  Did Loren sort of give up on that nanoparticle-covered-in-tethers idea or is that the ultimate point of the grant?  What sort of particles specifically do we want to look at?  All I really know is that we can look at stuff diffusing into mucus if I resuspend the mucus in precursor solution and crosslink it.  That's not good enough for this next grant proposal thing so we have to make hdyrogel windows which really isn't/wasn't going well. Plus put in nanoparticles so they jiggle around and we can measure the diffusion constant from the fluctuations.  Argh, I'm trying here.  I'm just so tired.
%
%Realistically, if I can just be totally honest for a bit, because none of this will go into the actual thesis, I don't think any of this will happen.  Probably not the PARP1 stuff, because I'm the only one who remembers to try and schedule meetings. Probably not the mucus stuff, because I can't get the hydrogel windows to work and the mucus auto-fluoresces in the wrong wavelength.  Certainly no more nuclear pore stuff ever.  Maybe FSFG will go on to have a lovely life as an NMR test peptide in stressed yeast cells, and that would be just fine.  No research project I've ever worked on, from freshman year of undergrad up until now, has ever had a publication result from it.  They have all quietly faded away into nothing.  Scott, who is not allowed to talk because he has a paper and a PhD but who is also a wonderful human being, says that it's because I'm drawn to really interesting and therefore really difficult problems.  Like a good scientist, I guess.  I gloomily say back to him that if everything I work on fails, there may be a common denominator.  Like that I'm a bad scientist.  

%Regardless.  That's not a debate we're going to settle soon I think.  The point is that what I think will really happen is that bits and pieces of this stuff will get picked up and incorporated into other projects.  Probably in really weird and interesting ways, because that's pretty much what Loren does, is combine bits of ideas in weird and interesting ways as fast and continuously as possible.  So I wrote this whole thing thinking of some anonymous future grad student ten years down the line who has been told, among five other `top priorities,' to make a hydrogel or analyze FRAP data or clone another version of NTF2 for the purposes of some whirlwind long-shot plan.  I hope they can use what I've done and that I've written it up well enough to take a bit of stress out of their day.  The only thing I can really predict right now is that I will/would be surprised to find out what bits of this work actually get picked up again and why.

