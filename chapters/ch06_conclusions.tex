\chapter{Conclusions and future directions}~\label{ch06_conclusions}

Living systems contain many examples of selective filters which control the transport of all manner of macromolecules.  In this work, we studied biofilters which allow rapid transport of highly specific targets, counterintuitively relying on binding interactions to promote rather than hinder flux.  These filters often contain similar features, such as transient, multivalent binding and binding to flexible, dynamic tethers.  In order to understand the presence of these features, we approached the problem through both modeling and experiment.  Bound-state diffusion proved to be a key parameter for selectivity.  While the theory and experimental setup were based on nuclear transport, they were not designed to exactly reproduce the nuclear pore, but to be applicable to a wider variety of biofilters.

We created a minimal model of the nuclear pore, containing very little beyond the diffusion of transport factors and their binding to FG Nups.  Interestingly, even this highly simplified model was able to reproduce the selectivity properties of the nuclear pore, predicting that binding could increase the flux of a protein through this material up to 300-fold over the flux of an identical inert protein.  The simplicity of our model made the key parameter clear: we could not reproduce the selectivity of the nuclear pore without allowing bound-state diffusion, that is, assuming that bound Nup - transport factor complexes were mobile within the pore.  Plausible mechanisms of bound diffusion in the nuclea pore include ethered diffusion arising from the disordered, flexible nature of the Nups, and inter-chain hopping of the transport factor due to its binding multivalency.  We modeled both of these mechanisms and found that they allow for significant bound mobility and therefore high selectivity.  These mechanisms may also be at work in other highly-specific, high-throughput biofilters.

After developing the minimal model of selectivity, we began developing a synthetic biofilter inspired by the nuclear pore with which to experimentally probe the effect of bound-state diffusion  on selectivity.  In order to capture the key features of selectivity, we designed hydrogels containing peptide tethers derived from FG Nups.  The transport factor NTF2 served as a test protein whose diffusion we could compare with a similarly-size but nonbinding counterpart.  Fluorescence recovery after photobleaching (FRAP) was used to determine the effective diffusion constants of both NTF2 and an inert protein, from which the bound diffusion constant was calculated. We measured a non-zero bound diffusion constant that is consistent with the predictions of our model, and tested the effect of varying Nup length on bound-state diffusion.

Additionally, we probed the aggregation behavior of an FG-Nup-derived peptide in several crowded conditions.  We used a fluorescent aggregation assay as well as NMR and fluorimetry to investigate differences in the aggregated state between crowding conditions, and found that even inert crowders which are widely used interchangeably show differences in the local chemical environments of the peptide.

Although a true hydrogel-based selective biofilter proved challenging to design, our nuclear-pore-inspired material can be used to measure the bound-state diffusion of proteins.  This parameter is likely important to a variety of problems involving rapid transport of highly-selective proteins.  The model and biomaterials developed here could be used to investigate these systems more generally in the future.  This work suggests that bound-state diffusion, particularly when resulting from transient, multivalent binding, may explain a number of unusual biological filters.

