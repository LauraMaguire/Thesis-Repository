\chapter{Conclusions and future directions}~\label{ch10_conclusions}
\section{Conclusions}
In this thesis, I have presented my advances to the understanding of particle
dynamics in crowded media and active matter.  For crowded media, I have shown
how soft obstacles affect tracer diffusion.  An obstacle's influence on tracer
dynamics is highly dependent on the stickiness of the obstacle, \text{i.e.},
the bound mobility.  Bound mobility can minimize the effects of inhibited motion
and anomalous subdiffusion.  The role of binding and bound motion was also
studied for transport of across biofilters.  I presented a minimal model in
terms of measurable parameters and studied the selectivity
of the nuclear pore complex.  Finally, I studied dynamical phases of an active
matter system.  I presented a macroscopic model, built from a microscopic
picture, using dynamical density functional theory.  I applied this theory to
study the banding instability of self-propelled needles. I contributed to and
independently wrote several software packages my various projects, summarized in
\appxref{appx:repos}.

For all of these systems, I developed models starting from microscopic
Brownian dynamics.  Studying particle dynamics  naturally lends itself
to two of the main tools discussed in this thesis: lattice Monte Carlo and
Brownian dynamics simulations.  I discussed how these tools were advanced and
applied to both crowded and active systems.  I also connected microscopic models
to macroscopic equations, \textit{i.e.}, nonlinear diffusion equations, through
DDFT\@.  In doing this, I've hoped to illustrate a general approach to solving
biophysics problems.  We start with a question or system of interest and try to
use our hammer, Brownian dynamics simulations (off or on lattice).  The
Langevin equations of motion, either with full interactions or making
approximations, are analyzed and refined until we have a minimal model that
describes some phenomena.  Deriving a theory for the continuum equations is a
harder task, but in doing so, larger systems and ensemble averaged behavior can
be studied. DDFT is one tool for obtaining these equations that has a direct
connection to the microscopic picture.

My final comment is a reflection on my graduate experience. All of my projects
have been substantially computational: involving collaborating on ideas,
distributing and receiving code, and working in shared repositories on projects.
From my experience, I really have come to value writing modular code that is
user-friendly and generalizable for other projects. For example, my repository
\textit{ActiveDDFT} can easily handle any external field or interaction (in the
mean-field or second virial approximation) with just a few lines of additional
code.  By implementing a mix of object-oriented and procedural code, I hope
that future students could easily use, adapt, and advance my software packages.
Gaining this perspective was a valuable lesson learned in graduate school. I
found that developing computational and problem solving skills, improving
communication, and gaining the confidence to tackle hard problems were the most
indispensable tools I gained in graduate school. 

\section{Future directions}

When thinking about future directions, I have some short term goals for my
research group and broader goals of the field. For my group, I wrote all
of my code with the intent that it can be applied to other related problems. I
believe there is further work that can easily be built onto my existing software
packages.

For the crowded media projects, including inter-particle interactions is what
I would look at next. For the lattice model, excluded interactions is not tough
to do. For a continuum model, a scaled particle
theory~\cite{helfand_theory_61} for steric interactions would be the first thing
to try. Long range interactions and external potentials could also be
implemented with DDFT to study the effects of charge. These improvements are
aligned with a broader goal for the field: directly modeling the temporal
evolution of experimentally measured density profiles in hydrogels.  No one
knows for sure how filters like the NPC work. I believe that accurately modeling
the temporal behavior of experimental results, \textit{e.g.}, TFs diffusing
across a hydrogel, will be the key to understanding these systems better.

My \textit{ActiveDDFT} repository, used in the active needle project, is well
equipped to handle other excess functionals. I believe studying active systems
with softer potentials, like a Gaussian core~\cite{lang_fluid_00,
dzubiella_meanfield_03} or soft shoulder~\cite{archer_quasicrystalline_13,
archer_softcore_15} (where DDFT has shown success), would be a natural next
step. For broader directions, relatively little work is being done to improve
models starting from microscopic interactions.  While models similar to our
active needle project claimed to see evidence of
flocking~\cite{baskaran_hydrodynamics_08}, we did not.  I think more work to
improve the microscopic functionals is necessary to describe the physics seen in
simulations. In a symmetry based model (which many groups use), it is not clear
how microscopic interactions can lead to macroscopic collective behavior. Also,
using a theory based on equilibrium systems may be suspect for active systems
far from equilibrium. One promising approach for nonequilibrium systems is
power functional theory~\cite{schmidt_power_13} which builds dissipation into
the model.

A final direction that is appealing is to unify my work on crowded and active
systems. One could imagine that obstacles could facilitate the self-assembly
process, and in reality, active biological systems can be
crowded~\cite{hofling_anomalous_13}.  The easiest way forward would be to
include obstacles in interacting active Brownian simulations. I
believe DDFT with external fields mimicking obstacles is feasible. 

After my Ph.D., I am off to industry to start a career in data science.  Machine
learning and statistical modeling are quickly advancing fields, and we are
living in an exciting time where technology is rapidly changing the world at an
unprecedented rate. I look forward to learning, growing, and continuing to solve
problems in a new field. If I help the machines take over, I am sorry.  But,
hey, at least we had a good run.
