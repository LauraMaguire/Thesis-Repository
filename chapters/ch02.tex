\chapter{Modeling}\label{ch02}

%Selective filters made of biopolymers are used in living and
%synthetic systems to control the localization and movement of
%molecules, nanoparticles, viruses and other organisms \cite{witten17}.
%These filters regulate access to genetic material (the nuclear pore
%complex, or NPC), cells (the pericellular matrix), tissues (the
%extracellular matrix), and organs (mucus). In addition to their
%protective role, polymeric biomaterials are physical barriers that can
%inhibit drug delivery. Microbial biofilms can sequester antibiotics in
%their pericellular matrix, hindering treatment of infections
%\cite{hoiby10}. The extracellular matrix and mucosal layers limit drug
%delivery in cancer and other diseases \cite{witten17}. How particle
%binding affects motion and filtering is unclear. Binding to the
%pericellular matrix facilitates uptake of nanoparticles by single
%cells \cite{zhou12}, and transport factors that bind to proteins in
%the NPC move rapidly through it \cite{strambio-de-castillia10}.  In
%contrast, binding inhibits the uptake of nanoparticles that bind to
%airway mucus \cite{schneider17, huang17, mastorakos15}. Many viruses
%minimize binding interactions, allowing human papillomavirus and human
%immunodifficiency virus to move nearly unrestricted in cervical mucus
%under certain conditions \cite{olmsted01,lai09}, although antibody binding 
%can slow these interactions \cite{chen14}.  Improved design of
%drug delivery vehicles and synthetic selective filters requires
%understanding what distinguishes these behaviors.  While particle
%size, charge, and binding interactions are known to affect filtering
%\cite{witten17}, the physical principles that underlie mobility and
%transport in polymeric biomaterials are not fully understood.
%
%Among these filters, the NPC is tuned for selective passage enabled by
%binding.  The NPC selectively filters molecular traffic between the
%nucleus and cytoplasm of eukaryotic cells, making it important for
%diverse processes including gene regulation and translation
%\cite{strambio-de-castillia10}. Transport occurs through the central
%channel, $\sim$50 nm in diameter and $\sim$100 nm long. The selective
%barrier filling the central channel is made from disordered proteins,
%the FG nucleoporins (FG Nups), which contain repeated
%phenylalanine-glycine (FG) motifs \figref{fig:cartoon}.  Transport
%factors (TFs) that directly bind to the FG repeats can cross the NPC
%and carry cargo with them \cite{strambio-de-castillia10}.  Transport
%through the NPC is remarkably fast, with pore residence times $\sim$10
%ms \cite{yang04}.  Binding between FG Nups and TFs shows
%diffusion-limited on-rates and transient binding of individual FG
%repeats to TFs \cite{milles15, hough15}.  How the FG Nups both block
%passage (of non-binding molecules) and facilitate passage (of binding
%molecules) is not fully understood, making the NPC an ideal system to
%dissect the principles of binding-controlled selective transport.

\begin{figure}[t!]
\centering
\includegraphics[width=17.8cm]{figs/ch02/fig1.pdf}
\caption{Schematics of the nuclear-pore complex and model. (A) The
  nuclear pore complex (gray) is filled with FG Nups (green polymers)
  that selectively passage transport factors that bind to FG Nups
  (blue) while blocking non-binding proteins (red). The central
  channel of the pore has length $L$. Protein concentration is high on
  the left (inlet) and low on the right (outlet).  (B) Selectivity
  quantifies the degree of selective transport through the pore. A
  non-selective pore with $S=1$ has the same flux for a transport
  factor as for a non-binding protein (top). A selective pore with
  $S>1$ has a larger flux for a transport factor than a non-binding
  protein (lower). (C) The bound diffusion coefficient quantifies the
  mobility of a bound transport factor.  A transport factor may be
  immobile (top) or mobile (lower) when bound. }
\label{fig:cartoon}
\end{figure}

Models of the NPC selective barrier have proposed that the FG Nups may
form an entropic brush \cite{rout00}, a dynamic hydrogel
\cite{ribbeck01, frey07}, an intermediate state between a brush and
gel \cite{vovk16}, or liquid droplets \cite{schmidt15}.  These
mechanisms may be modulated by spatial organization \cite{yamada10,
  ando14} and binding of TFs to multiple FG repeats \cite{lowe15,
  schoch12}.  Attempts to distinguish these models have been hindered
by the pore's small size, the redundancy and multiple copies of FG
Nups, and contradictory experimental results on FG Nups and TF binding
\cite{vovk16}. Some FG Nup fragments form less-dynamic hydrogels
\vitro\ \cite{frey07}, but remain highly dynamic within cells
\cite{hough15}. Molecular dynamics simulations find highly dynamic FG
Nups, though the degree and extent of motion depends on the affinity
of FG repeats for each other and for TFs \cite{pulupa17,
  vovk16}. Crowding and competition modulate affinity
\cite{tetenbaum-novatt10} and may contribute to selective transport
\cite{zilman07}.  However, the connection between the amino-acid level
behavior of the FG-TF interaction and macroscopic transport
selectivity remains unclear.  Here we address the central
contradiction of selective transport through the NPC: how does binding
of TFs to FG Nups within the pore increase the flux rather than
decreasing it \cite{bickel02, witten17}?

Using a biophysical model, we demonstrate that TF diffusion and
binding are sufficient for selective transport, as long as binding
only partially immobilizes TFs. Binding increases the local
concentration, and these molecules contribute to the flux if mobile.
Thermally-driven diffusion of TFs bound to flexible tethers gives
sufficient particle mobility to produce selectivity similar to
experimental measurements.  Tether flexibility also allows bound TFs
to hop between tethers, further enhancing selectivity.

\section{Simplified model of nuclear transport}

We consider a minimal model of the central channel of the NPC
containing FG Nups homogeneously anchored \figref{fig:cartoon}.  This
model is sufficiently general to describe the common features of a
range of biopolymer filters.  The NPC, unlike most other biopolymer
filters, has a wide capture area that may increase transport rates
\cite{pagliara14}.  In order to focus on basic principles of
transport, we neglect this effect.  A varying free energy landscape
along the axis of the NPC may play a role in selective transport
\cite{zilman07, tagliazucchi13, tu13, timney16}.  However, the NPC is
robust to deletion of all asymmetric Nups and many Nup combinations,
indicating that spatial variation in pore properties is not necessary
\cite{strawn04, zeitler04}.  Experiments \vitro\ with simplified,
homogeneous Nup composition produced selective transport
\cite{kowalczyk11, jovanovic-talisman09}.

\begin{figure*}[t!]
\centering
\includegraphics[width=\textwidth]{figs/ch02/fig2.pdf}
\caption{Flux through the pore and selectivity for TFs with varying
  bound mobility. (A) Flux as a function of time when TFs are immobile
  while bound, with varying binding affinity as in (B).  (B) Flux as a function
  of time when TFs are mobile while bound with $D_B = D_F$, with
  varying binding affinity.  (C) Selectivity as a function of
  dissociation constant with varying bound diffusion coefficient. }
\label{fig:transient}
\end{figure*}

Rapid transport requires TF-FG Nup binding, while a protein similar to
a TF but unable to bind FGs is excluded. Therefore, in our model we
compare two proteins that are identical, except that one binds FG Nups
and the other does not.  As a model TF, we consider nuclear transport
factor 2 (NTF2) \cite{ribbeck98}. NTF2 is small ($\sim$5 nm) relative
to the diameter ($\sim$50 nm) and length of the pore ($\sim$100 nm),
suggesting that passage of NTF2 does not require large-scale molecular
rearrangements that have been proposed for larger molecules
\cite{lowe15, frenkiel-krispin10}. Because of the small size of NTF2 we
neglect effects of steric crowding, which can enhance selectivity in a
transport model \cite{zilman07}.  NTF2 appears not to be actively
released from the pore, suggesting that selective transport is an
intrinsic property of the NPC \cite{mincer11, zilman07}, and in
contrast to actively released karyopherins \cite{lowe15, mincer11,
  gorlich96, gilchrist02}.

Transport through the NPC requires entry into the pore, passage, and
exit. In single-molecule measurements, most of the transport time is
spent in a random walk within the central channel \cite{yang04,
  tu13}. We therefore assume that entry and exit rates are determined
by binding kinetics (see Supporting Information, section
\ref{sec:entry} for the model when entry and exit are rate-limiting.)
The directional bias in TF transport is controlled outside the NPC
through a concentration difference between the nucleus and cytoplasm
generated by the Ran-GTP system \cite{riddick05}.  In our model, we
impose a fixed concentration difference across the pore.

%%%%% bare-bones description of model, not sure if I should try to alter

We consider a channel of length $L$ filled homogeneously with Nups
that separates two reservoirs \figref[A]{fig:cartoon}.  Within the
channel are free transport factor (concentration $T$), free FG Nups
($N$), and bound TF-FG complex ($C$), with total Nup concentration
$N_t= N+C$.  TF diffusion within the channel ($0<x<L$) is described by
the reaction-diffusion equations
\begin{eqnarray}
  \frac{\partial T}{\partial t} &=& -\kon T N+\koff C +D_F
       \frac{\partial^2 T}{\partial x^2},\label{eq:continuum_main} 
   \\ 
  \frac{\partial C}{\partial t} &=& \kon T N -\koff C + 
        D_B \frac{\partial^2 C}{\partial x^2} .
\label{eq:continuum_main_2} 
\end{eqnarray}
TF-FG interaction has on-rate constant $\kon$, off-rate $\koff$, and
dissociation constant $K_D =\koff/\kon$.  We include competition
between TFs for FG binding sites \cite{timney16}.  The
diffusion constants of free ($D_F$) and bound ($D_F$) TFs are
spatially constant. The fixed reservoir TF concentrations are $T_L$
(inlet, left) and 0 (outlet, right).

The flux of transport factor out of the pore
$J = - D_F \left. \partial T/\partial x \right|_{x=L}$. We numerically integrated
the full equations.  
%%%%%%

Because flux
measured in experiments is typically linearly proportional to TF
concentration \cite{timney06, schmidt15}, TF concentration likely
remains below binding saturation in the NPC. Therefore, we also solved
eqns.~(\ref{eq:continuum_main}, \ref{eq:continuum_main_2})
analytically in the low binding limit.  We define
the transport selectivity $S$ as the ratio of steady-state flux of a
binding versus a non-binding species \figref[B]{fig:cartoon}
\begin{equation}
  S =  \frac{J_{\rm binding}(t \to \infty)}{J_{\rm non-binding}(t \to \infty)} .
\end{equation}

\subsection{No selective transport occurs if bound TFs are immobile}
If TF-FG Nup binding immobilizes the TF, the bound-state diffusion
coefficient $D_B = 0$.  For immobile bound TFs, transport is not
selective: the steady-state flux $J = D_F T_L/L $ for both binding and
non-binding proteins, so $S = 1$ (figs.~\ref{fig:transient},
\ref{fig:linear-selectivity}).  The binding TF accumulates within the
pore, but its immobility means it does not enhance transport compared
to the non-binding case.  Notably, this effect is independent of
binding kinetics.  Prior to steady state, binding slows transport
\figref[A]{fig:transient}.  In systems such as airway mucus,
immobilization may increase the time available for degradation or
active clearance, consistent with the observation that binding tends
to inhibit selective transport in those systems \cite{schneider17,
  huang17, mastorakos15}.  This effect is related to the binding-site
barrier seen in antibody delivery to tumors \cite{juweid92}, and
observations that non-binding nanoparticles are often more effective
in drug delivery to tumors than binding particles \cite{witten17}.

Our model is related to the classic problem of molecular transport
through an oil membrane separating two aqueous reservoirs
\cite{schafer13}.  The relative concentration of a species just inside
the oil barrier to the concentration in water is called the partition
coefficient.  The steady-state flux through the membrane is directly
proportional to the partition coefficient (Supporting Information,
section \ref{sec:partition}, fig.~\ref{fig:oil-membrane}).  By
analogy, one might expect the TF-FG binding affinity to determine the
flux across the pore. However, binding is different from partitioning.
In systems where the increase in intra-pore concentration arises from
binding, the effective diffusion coefficient is typically inversely
proportional to the partition coefficient, making the flux independent
of binding affinity \cite{bickel02}.  This result led us to consider
whether TFs may be mobile while bound to FG Nups.




\subsection{Bound-state diffusion allows selective transport}
When bound TFs are mobile, selective transport occurs with a
selectivity up to 240 for a conservative set of parameters
(figs.~\ref{fig:transient}B,C, \ref{fig:linear-selectivity},
Supporting Information, section \ref{sec:param}).  Remarkably, this
selectivity is comparable to experimental measurements of NTF2 versus
GFP flux (Table \ref{table:NTF2-flux}).  The interplay between binding
kinetics and diffusion leads to an optimal dissociation constant
$\sim$1 $\mu$M for maximum selectivity \figref[C]{fig:transient}.
Selectivity decreases for high $K_D$ because binding is too weak to
significantly increase TF concentration in the pore.  For low $K_D$,
tight binding causes the concentration of bound complexes to become
approximately constant across the pore. Because diffusive flux is
driven by a concentration gradient, this washing out of the gradient
by tight binding decreases flux and selectivity.

\begin{table}[b!]
  \caption{Comparison between experimental results for NTF2 and GFP
    (a similarly-sized non-binding protein) and model
    predictions. Flux measured in units of molecules per pore per
    second.}
    \label{table:NTF2-flux}
    \begin{tabular}{p{2.1cm}p{1.2cm}p{1.7cm}p{0.9cm}p{1.6cm}p{0.8cm}}
      Method & Cell type & Species & Flux & Selectivity & Notes\\
      \hline
      OSTR & \textit{Xenopus} & \makecell[cl]{NTF2\\GFP} & \makecell[cl]{91--123\\3.3--3.8} & 24--37 
                         &\cite{siebrasse02}
      \\
      OSTR & \textit{Xenopus} & \makecell[cl]{NTF2\\GFP} & \makecell[cl]{47.3\\1.1} & 43 &  \cite{kiskin03}\\
      \makecell[cl]{Permeabilized \\ cells}  & HeLa &
                                                    \makecell[cl]{NTF2\\GFP} & \makecell[cl]{250\\2} & 125 & \cite{ribbeck01}\\
      Model & -- & \makecell[cl]{Binding\\Non-binding} & \makecell[cl]{2--480\\2} & 1--240 & \makecell[cl]{This\\work}\\
    \end{tabular}
\end{table}

Our model predicts that selectivity is increased by increasing binding
on-rate constant $\kon$ \figref{fig:parameter-variations}. Consistent
with this, the on-rate constants of TF-FG Nup interactions have been
measured to be diffusion limited \cite{milles15, hough15}.  Large
$\kon$ makes transport more selective because fast binding kinetics
relative to diffusive motion are necessary to maintain steep
concentration gradients within the pore. High FG Nup concentration (as
measured experimentally) leads to large $N_t$ and low $D_F$, both of
which increase selectivity.  Decreasing $D_F$ or increasing the length
of the pore both reduce the magnitude of the flux and increase
selectivity (figs.~\ref{fig:parameter-variations},
\ref{fig:parameter-variations-abs-flux}). Therefore, varying TF free
diffusion coefficient and pore length involves a trade-off between
transit time and selectivity.


\section{Mechanisms of  bound transport factor mobility}

Our result that bound-state diffusion is required for selective transport raises a mechanistic question: how can TFs move while bound to FG Nups? Here we consider two experimentally based mechanisms: movement of the bound TF due to the intrinsic flexibility of the FG Nups \cite{patel07} and multivalent binding that allows hopping of TFs between neighboring Nups \cite{raveh16}.

%%%%%%% This portion is good I think
\section{Bound mobility through tethered diffusion}
%I wrote the math for this section of the paper
One possible mechanism of bound-state diffusion within the nuclear pore is tethered diffusion.  FG Nups, as disordered proteins, are flexible and highly dynamic \cite{lim07, milles14, hough15}. It is not clear whether they form polymer brushes or crosslinked hydrogels within the nuclear pore, but in either case tethered diffusion remains a viable mechanism of bound diffusion.  In the case of a polymer brush, one end of an FG Nup is anchored to the NPC scaffold, but the other end is free, affording mobility to a bound transport factor.  If FG Nups are crosslinked, the effective length of the flexible tether will be shorter, but the same principle of tethered diffusion will apply.

Flexible polymers behave as entropic springs \cite{howard01} if they are not highly stretched. Therefore, a bound TF diffuses while attached to a spring-like tether, which can be represented as diffusion in a harmonic potential well \figref[A]{fig:tethers}.  The width of the harmonic well is related to the length of the flexible domain.  The effective length is the full FG Nup length if the FG Nups are not crosslinked, while the effective length is reduced if they are crosslinked or entangled\cite{ribbeck01}.  

In order to calculate the bound diffusion coefficient of the TFs, an averaging procedure is followed.  The diffusion is assumed to be Fickian, which is a reasonably good though not perfect assumption. (See discussion in Sec.~\ref{sec:fickian}.)  In the Fickian diffusion case, the diffusion coefficient is proportional to a mean-squared displacement (MSD) divided by time.  We calculate the mean binding lifetime $\tau$ and the MSD corresponding to this ``typical'' binding event and divide them.

To begin, note that the duration of a binding event follows the exponential distribution 
\begin{equation}
\rho (t) = \exp(-t/\tau)/\tau\,,
\end{equation}
where $\tau = 1/\koff$ is the mean binding lifetime.

Next, the positional probability density of a bound TF is 
\begin{eqnarray}
P(x,t) &=& e^{-\frac{x^2}{2 \alpha(t)}}/\sqrt{2\pi \alpha(t)}\,,\\
\alpha(t) &=& (1-e^{-2kD_F\beta t})/(k\beta)
\end{eqnarray}
 where $k$ is the spring constant of FG Nup tethering and $1/\beta = k_BT$ is the thermal energy \cite{doi88}.  The center of the well is set at $x=0$.

The mean-squared displacement (MSD) of the TF as a function of time is calculated, as any expected value, with the integral
\begin{equation}
\ev{x^2(t)} = \int_{-\infty}^{\infty} P(x,t) x^2 dx = \alpha(t)\,.
\end{equation}

Finally, the typical TF MSD during a binding event can be determined by evaluating
\begin{equation}\label{eq:sho}
  \overline{\ev{x^2}} = \int_0^{\infty} \rho(t') \ev{x^2(t')} dt' = \frac{2D_F L_c
    \ell_p}{L_c \ell_p \koff+ 3D_F}\,. 
\end{equation} % Mike realized this was actually a pretty easy integral - should I credit him here?
Here we assume that the spring constant is that of a worm-like chain polymer $k = 3/(2\beta L_c \ell_p)$, where $L_c$ is the contour length and $\ell_p$ the persistence length \cite{howard01}.

Combining these results, the one-dimensional bound diffusion coefficient is
\begin{equation}\label{eq:dbound}
  D_B \approx \frac{\overline{\ev{x^2}}}{2\tau} = \frac{D_F L_c \ell_p
    k_\un{off}}{L_c \ell_p k_\un{off} + 3D_F} =
  \frac{D_F}{1+3\frac{D_F}{D_{P}}}.  
\end{equation}
Here $D_P = L_c \ell_p k_\un{off}$ controls the bound-state diffusion coefficient: higher $D_P$ corresponds to a lower constraint of the TF by the tether and greater bound mobility. Bound mobility increases with increasing chain length and persistence length, or decreasing binding lifetime. When $D_P$ is large ($D_F/D_P\ll1$), $D_B$ approaches $D_F$, since the long chains barely affect TF motion during the short binding event. For small $D_P$ ($D_F/D_P\gg1$), TF motion is inhibited by a short tether, giving $D_B\approx D_P/3\ll D_F$.  

\begin{figure}
\centering
\includegraphics[width=\textwidth]{figs/ch02/fig3.pdf}
\caption{(A) Schematic of the flexible tether model of bound-state
  diffusion. FG Nups are treated as entropic springs that constrain
  the motion of TFs more (top and center left, longer FG Nup) or less
  (top and center right, shorter Nup), which corresponds to changing
  width of the harmonic potential well (lower).  (B) Ratio of bound to
  free diffusion coefficient as a function of dissociation constant,
  with varying polymer length in the tethered-diffusion model.  (C)
  Selectivity as a function of $K_D$, with varying polymer length in
  the tethered-diffusion model.  Selectivity calculated by Mike Stefferson.}
\label{fig:tethers}
\end{figure}

Physiological values can be estimated for all of the relevant parameters.  Disordered proteins are relatively flexible, with persistence lengths around $\ell_p \approx 1$ nm \cite{receveur-brechot12}.  The contour lengths of the disordered regions of FG Nups are in the range $L_c\approx$ 100--280 nm (250--700 amino acids long \cite{patel07} with a contour length per amino acid $\approx 0.4$ nm).

Experimental evidence suggests that the on-rate constant of Nup-transport factor binding is diffusion-limited, with $\kon = 10^{-3}\mu \tx{M}^{-1} \, \mu \tx{s}^{-1}$ \cite{milles15, hough15}.  The off-rate constant is given by $\koff = \kon K_D$.  Measured values of the dissociation constant $K_D$ span several orders of magnitude \cite{things}; therefore the off-rate constant is not well-determined.  Figure~\ref{fig:tethers}B and C show the bound diffusion coefficient and selectivity for a range of $K_D$ values spanning those measured experimentally, with a fixed, diffusion-limited on-rate.

Finally, an estimate of the free diffusion constant of NTF2 moving within the nuclear pore is needed.  This value has not been directly measured, but it was estimated at $D_F = 0.12 \ \mu\tx{m}^2$/s using a similar value for a non-binding protein \cite{ribbeck01}.  As would be expected, this diffusion constant is smaller than that of a karyopherin in the nucleus, which has been estimated at $D_F =1 \ \mu\tx{m}^2$/s \cite{cardarelli10}.

\begin{SCfigure}
\centering
\includegraphics[width=0.5\textwidth]{figs/ch02/chain-comparison.pdf}
\caption{Selectivity as a function of dissociation
  constant in the tethered diffusion model, varying Nup contour length $L_c$ and total Nup concentration $N_t$.  The product $L_cN_t$ is held constant.\\}
\label{fig:chainComparison}
\end{SCfigure}

Using these parameters, along with those described in earlier sections, Mike Stefferson calculated the selectivity due to tethered diffusion for several tether lengths (Fig.~\ref{fig:tethers}C).  Selectivity was also calculated for two values of the total Nup concentration $N_t$, mimicking the possible effect of Nup crosslinking within the pore (Fig~\ref{fig:chainComparison}.  The produce of contour length and Nup concentration $L_cN_t$ was held constant, and selectivity calculated for a long Nup length of 100 nm as well as a shorter length of 25 nm, reflecting the possibility of crosslinking.  Corresponding total Nup concentrations of 4.7 and 18.8 nm, respectively, were determined from an estimate of the number of TF binding sites ($800$), and the volume of a cylinder of diameter $60\ \nm$ and length $L = 100$ nm. 

 Using these realistic parameters, selectivity can reach 200-300, a large flux enhancement for TFs over nonbinding proteins.

\section{Bound mobility through inter-chain hopping}

Another possible mechanism of bound-state diffusion is inter-chain hopping enabled by multivalent binding interactions.  All known transport factors have at least two hydrophobic binding pockets which bind to FG motifs, and some have many more \cite{something?}.  FG Nups in turn each possess many FG motifs, leading to a highly multivalent binding interaction.  This feature allows a transport factor to bind to multiple Nups at once, moving between them with a hand-over-hand or sliding motion without ever fully unbinding \cite{raveh16, tetenbaum-novatt12}.  While binding to multiple FG motifs on the same Nup will not lead to bound diffusion, inter-chain hopping will cause the origin site of tethered transport factor diffusion to change over time.  In order to understand the effect of hopping on the overall bound diffusion constant, we model a TF that undergoes tethered diffusion when bound to an FG Nup and hops between neighboring, randomly distributed tethers \figref{fig:hopping}.

\begin{figure*}
\centering
\includegraphics[width = \textwidth]{figs/ch02/fig4.pdf}
\caption{(A) Schematic of the inter-chain hopping model of bound-state diffusion. FG Nups are treated as entropic springs that constrain the motion of TFs, and inter-chain hopping allows a TF to move from one FG Nup (top and center left, green Nup) to another (top and center right, red Nup) without unbinding, which corresponds to switching from one harmonic potential well to another (lower). (B) Ratio of bound to free diffusion coefficient as a function of dissociation constant, with varying hopping rate in the inter-chain hopping model.  (C) Selectivity as a function of $K_D$ with varying hopping rate. FG Nup contour length $L_c = 40$ nm in (B, C). }
\label{fig:hopping}
\end{figure*}

In our simulation of TF motion with hopping between FG Nups while bound, we represented each FG Nup as an entropic spring (i.e. as a harmonic potential well).  Well positions were randomly chosen from a uniform distribution, with the exception that we always placed one well at the starting position of the TF.  The particle (the TF) started the simulation bound to this FG Nup, and remained bound throughout the simulation.  While bound to one FG Nup, the TF diffused within the harmonic well representing that FG Nup. We recorded the position and mean-squared displacement of the TF from its starting location, which we then used to determine a bound diffusion coefficient, as described in more detail below.  The TF could hop between tethers by changing which well it moved in.

% I wrote all of the sections below, describing the hopping simulation, for the paper.
\subsection{Diffusion in a potential well}

The TF moved in the harmonic potential of the FG Nup according to Brownian dynamics. At each timestep, the TF position was updated using a force-dependent diffusive step \cite{blackwell17}.
\begin{equation}
  x(t+\delta t) = x(t) + \frac{F}{\Gamma} \delta t + \delta x\,,
\end{equation} 
where $F$ is the force acting on the particle, $\Gamma$ is the drag coefficient, $\delta t$ is the timestep, and $\delta x$ is a random Brownian step drawn from a Gaussian distribution with variance $\sigma^2 = 2 D \delta t$. The drag coefficient of a spherical particle at low Reynolds number is given by Stokes' Law as $\Gamma = 6 \pi \eta r$, where $\eta$ is the fluid's viscosity and $r$ is the sphere's radius.  This result can be combined with the Einstein relation $D = k_B T / (6\pi \eta r)$ to give
\begin{equation}
\Gamma= \frac{k_B T}{D}\,.
\end{equation}
 
The force $F = -k\Delta x$, where $k$ is the spring constant of the FG Nup and $\Delta x$ is the displacement of the particle from the Nup attachment point.  We model the FG Nup as a worm-like-chain at small extension, so that $k = 3 k_B T/(2\ell_pL_c)$, where $\ell_p$ is the tether persistence length and $L_c$ is the contour length.  Then 
\begin{equation}
  x(t+\delta t) = x(t) - \frac{3 D \Delta x \delta t}{2\ell_p L_c }+
  \delta x = x(t) - D K \Delta x \delta t+ \delta x\,,
\end{equation}
where $K$ is the normalized spring constant $K = k/k_B T = 3/(2 \ell_p L_c)$.

\subsection{Hopping probability}

We designed the hopping probability $\Phop$ in order to satisfy the principle of detailed balance.  During every iteration of the simulation, we picked an FG Nup at random from a list of the $M$  Nups near enough to have a reasonable probability of hopping. TF hopping to the new FG Nup was attempted with success probability
\begin{equation}
\Phop = r_\tx{hop} M \delta t e^{-\Delta G /2}.
\end{equation}
Here the base hopping rate $r_\tx{hop}$ is a dimensionless input parameter, and the change in free energy (in units of $k_BT$) between the current Nup and the proposed new Nup is
\begin{equation}
  \Delta G = \frac{1}{2} K (x-x_\tx{new})^2 - \frac{1}{2} K (x -
  x_\tx{cur})^2,
\end{equation}
where $K$ is the normalized spring constant, $x$ is the particle's current position, $x_\tx{cur}$ is the anchor location of the Nup to which the particle is currently bound, and $x_\tx{new}$ is the anchor location of the proposed new Nup. Note that when a hop succeeds, the energy landscape changes to that of the new Nup, but the TF's position does not change during the hop.  There is no upper bound on $\Phop$, but we adjusted the timestep to ensure that $\Phop$ was greater than
unity no more than 0.5\% of the time that a hop was attempted.

\subsection{Mean-squared displacement and diffusion coefficient calculation}
We ran each simulation for $10^7$ time steps with $\delta t = 0.01$ $\mu$s, and recorded the particle's position every 100 time steps.  We calculated the mean-squared displacement $\ev{x^2}$ (MSD) of the TF and averaged it over 100 runs \figref[A]{fig:integrand}.  We then computed
\begin{equation}
\rho_\tx{MSD} (t)= \ev{x^2(t)} \rho(\koff,t) = \koff \ev{x^2(t)}
e^{-\koff t}, 
\end{equation}
as shown in fig.~\ref{fig:integrand}B, and numerically integrated the distribution in time. We determined the bound diffusion coefficient from the typical MSD-per-binding-event $\overline{\ev{x^2}}$ using
\begin{equation}
D_B = \frac{\koff \overline{\ev{x^2}}}{2}. 
\end{equation}   
Here, the factor of $1/2$ is appropriate because we consider a one-dimensional random walk.

\begin{figure}[h!]
\centering
\includegraphics[width=0.7\linewidth]{figs/ch02/integrand-example-plots.pdf}
\caption{(A) Examples of mean-squared displacement (MSD) of a simulated TF in the inter-chain hopping model, with varying hopping rate.  (B) Examples of MSD distributions $\rho_\tx{MSD} (t)$ used in estimating the diffusion coefficient, with varying unbinding rate. Tethers have 40 nm contour length; other parameters are as discussed in the text.}
\label{fig:integrand}
\end{figure}

\subsection{Bound diffusion and selectivity from hopping simulation}

Upon calculating the bound diffusion constant using the hopping simulation described above, it was clear that inter-chain hopping could lead to relatively large bound diffusion constant, with a corresponding increase in selectivity (Figs.~\ref{fig:hopping}, \ref{fig:partitioningB}, \ref{fig:partitioningC}).  In the limit of a hopping rate of zero, the tethered-diffusion-only result is recovered, as anticipated.  Hopping most enhances selectivity when the Nup length or dissociation constant are small.  This is the regime where tethered diffusion is limited, corresponding to crosslinked Nups within the nuclear pore.  If the pore is highly crosslinked, binding multivalency may be essential to selectivity.

\begin{figure}[h]
\centering
\includegraphics[width=0.7\linewidth]{figs/ch02/hopping_lc4-fig.pdf}
\caption{Bound diffusion and selectivity as a function of dissociation
  constant, with varying hopping rate for FG Nups with $L_c = 4$ nm.}
\label{fig:partitioningB}
\end{figure}
\begin{figure}[h]
\centering
\includegraphics[width=0.7\linewidth]{figs/ch02/hopping_lc12-fig.pdf}
\caption{Bound diffusion and selectivity as a function of dissociation
  constant, with varying hopping rate for FG Nups with $L_c = 12$ nm.}
\label{fig:partitioningC}
\end{figure}
\begin{figure}[h]
\centering
\includegraphics[width=0.7\linewidth]{figs/ch02/hopping_lc120-fig.pdf}
\caption{Bound diffusion and selectivity as a function of dissociation
  constant, with varying hopping rate for FG Nups with $L_c = 120$ nm.}
\label{fig:partitioningD}
\end{figure}

%%%% end of portion I've checked


\section{NPC parameters}
\label{sec:param}

We used the FG-filled pore length $L = 100\ \nm$
\cite{frenkiel-krispin10, maimon12}; $L=30\ \nm$ gave similar results (data not shown).  Total FG Nup concentration was determined from an estimate of the number of TF binding sites ($800$), and the volume of a cylinder of diameter $60\ \nm$ and length $L$.  


 The barrier imposed by the FG Nups to a TF and
its non-binding counterpart is incorporated into the effective TF
concentration at the edge of the gel.  We estimate this barrier
$\approx 1.5\ k_B T$ for an NTF2 sized molecule \cite{timney16}.  We
estimate the cytoplasmic concentration of NTF2 is 5 $\mu$M.  Then
$T_L= 5 \times e^{-1.5} \mu \tx{M} = 1 \tx{\ \mu M}$.  The binding
affinity of TFs for FG Nups within the NPC is not well known, and
involves a complex interplay between site-specific affinity and change
in polymer conformations \cite{vovk16}.  Therefore, we consider a
range that includes most measurements and estimates of $K_D$ between
10 nM and 10 $\mu$M \cite{pyhtila03, gilchrist02, tetenbaum-novatt12-1,
  milles15, timney16, vovk16}.  

\section{Fickian diffusion}
\label{sec:fickian}
Talk about how diffusion is almost but not quite Fickian, and show the anomalous diffusion plots (made with Mike's script) for various parameters. Talk about timescales and how this might affect the results.

\section{Discussion}
A key puzzle of the NPC is how transport-factor binding allows rapid transport through the pore.  Binding typically immobilizes the bound particle, and so the increase in concentration resulting from binding does not, in general, result in increased flux. The biophysical theory
we developed includes diffusion of TFs due to thermal fluctuations,
binding to polymeric tethers, and the hopping of bound species between
these tethers.  Thus we identified principles of selective transport
resulting from binding \figref{fig:cartoon}, emphasizing that
bound-state mobility is essential for selective transport
\figref{fig:transient}.  Binding increases the local concentration,
and any bound mobility increases the flux.  We characterized two
mechanisms to obtain bound-state mobility and found that
thermally-driven diffusion of TFs bound to flexible tethers and rapid
binding kinetics \cite{hough15, milles15} allow TF mobility, leading
to selectivity similar to that observed experimentally
\figref{fig:tethers}.  In addition, tether flexibility enables
multivalent bound particles to hop between binding regions
\figref{fig:hopping} \cite{lowe15, schoch12}, further enhancing
selectivity. Mobility of bound or partitioned molecules occurs in many
biological contexts, suggesting that the mechanisms we study here may
be broadly applicable \cite{stefferson17, braga07}.


Our model for selective transport by tethered diffusion generalizes to
a range of FG-FG interactions \cite{vovk16}, if we decrease the
effective chain length $L_c$ for cohesive FG Nups.  For short chains
the selectivity simply due to chain flexibility is modest, suggesting
that other mechanisms, like hopping, may be important.  Our model
suggests that transient cross-linking of FG repeats proposed to occur
within the pore may serve to increase the viscosity and therefore the
selectivity. Crosslinks need not be actively melted by TFs to enhance
selectivity \figref{fig:parameter-variations}.
  
Our model provides a quantitative tool to evaluate selective
transport. Materials formed \textit{in vitro} by spontaneous
self-assembly of FG Nups \cite{frey07} or transient crosslinking by
alpha-helical peptides \cite{kim15} show strong selective
\textit{entry}.  Using published data, we predicted whether these gels
also showed selective \textit{transport} (table \ref{table:Gorlich}).
Most synthetic gels are predicted to have $S<10$, less than the
selectivity of NTF2 in cells (table \ref{table:NTF2-flux}). The
predicted selectivity of one hydrogel is $S\approx200$, apparently the
most selective synthetic gel to date \cite{frey07}.


\subsection{Overcoming the limitations of binding}
Binding, even in the presence of bound-state motion, limits
selectivity.  Biological systems appear to have developed strategies
to avoid this, for example, by using true partitioning.  Lipid domains
in complex membranes partition proteins \cite{simons11}.  Membraneless
organelles spontaneously assembled from low-complexity proteins and
nucleic acids can localize a molecule without immobilizing it
\cite{brangwynne15}.  Because membraneless organelles are fluid, the
constraints imposed in our NPC model by binding are released.  Our
work thereby suggests a benefit of phase-separated droplets to cells:
they provide significantly higher selectivity than can occur with
immobilizing binding. This may be especially important for spatially
complex assemblies \cite{feric16}.

Though we show it is not necessary, the active dissolution of
polymeric biomaterials has been proposed to occur in the NPC
\cite{ribbeck01}. This strategy is used by \textit{Helicobacter
  pylori} to penetrate the gastric mucus \cite{celli09}. Because the
particularly dense extracellular matrix of solid tumors blocks the
motion of particles, especially larger nanoparticles, ECM dissolution
has been used to enhance drug delivery \cite{zhou13}. Unfortunately,
this approach may not be universally applicable: breaking down the ECM
surrounding tumors may promote cancer metastasis \cite{miao15}.

\subsection{Design principles of selective transport by binding}

Filtering by polymeric biomaterials occurs in many systems for
particles of different sizes: for example, nutrients reach our
intestinal walls while larger molecules are excluded.  However,
controlling the selective transport of similarly-sized molecules by
tuning specific interactions has proven elusive. In drug delivery
applications, inert nanoparticles are typically more effective at penetrating
extracellular spaces and reaching their cellular targets
\cite{witten17}. Because biopolymer filters are the first point of
contact of nanoparticles used for drug delivery, specific targeting of
transport through mucus may enhance the effectiveness of drug
delivery. If NPC-like bound mobility as described in our model could
be achieved in these systems, it would increase the rates of
transport and drug delivery. 
