help and laughs\chapter{Introduction}~\label{ch01_introduction}

Interactions, either between particles or with the environment, play a crucial
role in the dynamics of biological systems.  For example, there are biological
polymeric filters which facilitate the transport of molecules, nanoparticles,
viruses and other organisms through interactions between the polymers and
particles~\cite{witten_particle_17}.  In biological crowded media, obstacles can
inhibit motion and introduce anomalous subdiffusion~\cite{hofling_anomalous_13}.
This has implications for the motion of lipids or macromolecules in
membranes~\cite{schutz_singlemolecule_97,schutz_properties_00,
  nicolau_sources_07, weigel_ergodic_11, javanainen_anomalous_13,
  krapf_mechanisms_15, jeon_protein_16, sadegh_plasma_17} and the cell
interior~\cite{stylianidou_cytoplasmic_14}.  Active matter systems,
\textit{e.g.}, a bacterial swarm~\cite{zhang_collective_10,
  cisneros_dynamics_11,thutupalli_directional_15}, can exhibit an array of
exotic phenomena: dynamical phases, self assembly, and collective
motion~\cite{marchetti_hydrodynamics_13} because of the interplay between
particle-particle interactions and nonequilibrium activity.  Understanding how
microscopic interactions influence particle dynamics, transport, and macroscopic
phases could assist in designing and tuning these systems.

The commonality between these crowded and active systems is Brownian motion.  A
particle is considered Brownian if it undergoes random motion in an environment,
typically called a heat bath.  The heat bath exerts two forces on the Brownian
particle: a random fluctuating force and frictional
drag force~\cite{zwanzig_nonequilibrium_01}. Note, the term heat bath, solvent, and
solution will be used interchangeably to describe the surrounding fluid
environment.  Brownian motion was first reported by Robert Brown in 1827 when he
observed the random motion of pollen in water.  Einstein's interpretation that
the phenomenon could be described by water molecules  bumping into the pollen
provided further evidence of the existence of atoms and
molecules~\cite{einstein_uber_05}. If we knew all the trajectories of a solvent
particles, we could in principle calculate the motion of the Brownian particle.
However, these degrees of freedom are intractable.  Instead, we approximate the
effect of the solvent particles as a random fluctuating and damping force.  The
equation which describes this motion is the Langevin
equation~\cite{zwanzig_nonequilibrium_01}. In most cases, we are interested in
the overdamped limit, which gives us Brownian dynamics (BD).

The study of Brownian dynamics is ubiquitous in physics: diffusion of holes and
electrons in semiconductors~\cite{ashcroft_solid_76}, Doppler cooling of neutral
atoms~\cite{lett_optical_89}, and motion of macromolecules in biological
systems~\cite{berg_random_93}.  The primary focus of this thesis is analyzing
different Brownian systems found in biology.  I will examine the role of binding
and bound motion in crowded environments, transport of proteins across the
nuclear pore complex (NPC), and dynamical phases in active matter systems.  I
will use Brownian dynamics simulations, lattice Monte Carlo models, kinetic
Monte Carlo, and numerical analysis of partial differential equations and
describe how I implemented and advanced these tools.

In this thesis, I will begin by first discussing various Brownian systems and
providing the essential theory.  I will then discuss a lattice Monte Carlo
model, which studied the effects of binding and bound motion on diffusion in a
crowded environment~\cite{stefferson_effects_17}; results that have
implications for various biological systems.  Next, I will discuss transport
across biofilters like the nuclear pore complex.  I will present a minimal
reaction diffusion equation that can predict the selectivity (\textit{i.e.},
what proteins can and cannot easily pass through) of the NPC in terms of
measurable parameters.  I then move to active matter systems: nonequilibrium
systems which can convert some source of stored energy into (kinetic)
motion~\cite{marchetti_hydrodynamics_13}.  I will discuss how to derive
macroscopic equations of motion starting from microscopic interactions using
dynamic density functional theory (DDFT)~\cite{marconi_dynamic_99,
  marconi_dynamic_00, archer_dynamical_04, archer_dynamical_05}, and how a
pseudospectral theory was implemented to numerically evaluate them.  Finally, I
will apply this theory to a system of self-propelled hard needles and conclude.


%%%%%%%% SECTION: BROWNIAN SYSTEMS  %%%%%
\section{Brownian dynamics}

The equation of motion of a particle in a one dimensional heat bath is given by
the Langevin equation~\cite{zwanzig_nonequilibrium_01} 
%
\begin{equation}
  \label{eqn:langevin}
   m \ddot{x} = -\frac{d U(x)}{dx} -\zeta \dot{x} + \xi(t),
\end{equation}
%
where $x$ is the position, $\zeta$ the drag coefficient, $U(x)$ the potential
energy from external fields, $\xi$ the random uncorrelated fluctuating force,
or noise, and the overdot means derivative with respect to time. The  average
properties of the noise are
% Noise
\begin{gather}
  \langle \xi( t ) \rangle = 0, \\
  \label{eqn:noise}
  \langle \xi( t ) \xi( t') \rangle 
  = 2 \zeta k_B T \delta(t-t'),
\end{gather}
%
where $k_B$ is the Boltzmann factor, $T$ the temperature, $\delta$ the Dirac
delta function, and $\ev{\ldots}$ the ensemble average over all noise
realizations.  Here, we assumed that the drag is homogeneous (constant in
space).  The Langevin equation is Newton's second law with
forces from gradients in potential energy and the drag
and random force provided by the heat bath.  The drag and random force are
connected through the drag coefficient (\eqn~\ref{eqn:langevin}
and~\ref{eqn:noise}).  This is a consequence of the fluctuation-dissipation
theorem~\cite{zwanzig_nonequilibrium_01}, which is derived for Brownian motion in
\appxref{appx:fluct_dissipation}.

To understand the overdamped limit, assume the particle experiences a constant
force $f$ from an external field and temporarily ignore the random force to
determine the behavior of the average velocity.  Solving the linear
ordinary differential equation for the velocity, we get
%
\begin{equation}
  \dot{x}(t) = f / \zeta + e^{-\zeta t/m} 
    \left[ \dot{x}(0) - f / \zeta \right].
\end{equation}
%
The particle's velocity exponentially decays with rate $\zeta /m $ from its
initial value to a steady state value $ f / \zeta $.  Now, if we are interested
in time scales larger than $ \tau_r = m / \zeta $, the velocity is constant.
These time scales can be extremely small, $ \tau_r \sim 10^{-7}$--$10^{-9}
\sec$, in colloidal and biological systems~\cite{dhont_introduction_96,
  berg_random_93}.  Brownian dynamics is this overdamped limit: the velocity
decays so quickly to a steady state value that we can drop the inertial terms
from the Langevin equation.  In the overdamped limit, the Langevin equation
describing Brownian dynamics~\cite{doi_theory_88, zwanzig_nonequilibrium_01} is
%
\begin{equation}
  \label{eqn:brownian_1d}
  \dot{x} = \zeta ^{-1} \left[  -\frac{d U (x)}{dx } + \xi(t) \right].
\end{equation}
%
\eqnref{eqn:brownian_1d} resembles a biased random walk, allowing us to discover
the connection between the Langevin equation and diffusion.

%%%%%%%% SECTION: RANDOM WALKS 101  %%%%%
\section{Random walks 101}
The random fluctuating force leads to a connection between Brownian dynamics and
a discrete random walk. By discrete, I mean that the particle will take a random
step $\Delta x$ at a rate $ 1 / \tau $. On average, $\langle \Delta x \rangle =
0$ and $\langle \Delta x^2 \rangle = l^2$ where $l$ is the characteristic step
length scale.  After $N$ steps, the particle's total displacement is a sum over
$i$ steps
%
\begin{equation}
  x_N - x_0 = \sum^N_{i=1} \Delta x_i.
\end{equation}
%
This equation is at the heart of the lattice Monte Carlo models and BD
simulations we will discuss later.  Let's imagine repeating this experiment
several times.  In doing so, we hope to say something meaningful about the
average behavior of the motion by exploiting the properties of the random step.
Upon averaging over experiments, the mean squared displacement (MSD) is
%
\begin{equation}
  \langle x_N - x_0 \rangle^2 = 
  \left \langle \sum_i^N \Delta x_i\sum_j^N \Delta x_j \right \rangle 
  = N l^2  = 2 D t,
\end{equation}
%
where $D = \frac{l^2}{2 \tau}$. Here, we used the fact that steps at different
times are uncorrelated $ \langle \Delta x_i \Delta x_j \rangle = \delta_{ij} l^2
$ and that $N = \frac{t}{\tau}$ where $t$ is the total time of the random walk.
$D$ is the diffusion coefficient; it connects the measurable spread of a random
walk, \textit{i.e.}, $\langle x^2 \rangle$, to time.  For arbitrary dimension
$d$, the MSD is
%
\begin{equation}
  \label{eqn:msd}
  \langle r^2(t) \rangle = 2dDt,
\end{equation}
%
where now $ D = \frac{l^2}{2d\tau} $, $\tau$ the step frequency along any
dimension, and $r$ the total displacement, $r^2 = \sum_{i=1}^d x_i^2$.

A brief comment on what we mean by averaging $\langle \ldots \rangle$.
Typically, this is called an ensemble or noise average.  This means repeating
the experiment with the same initial conditions and averaging the results.  By
doing this, we are exploiting the average properties of the noise.

Now that we explained the connection between random walks and the diffusion
coefficient, we can return to \eqnref{eqn:brownian_1d}. In the absence of
interactions and fields, the overdamped Langevin equation is
%
\begin{equation}
  \dot{x} = \zeta^{-1} \xi(t).
\end{equation}
%
Upon integration with respect to time, we get
%
\begin{equation}
  \label{eqn:brownian_int}
  x(t) - x(0) =  \int_0^t \zeta^{-1} \xi(t') \dif t'.
\end{equation}
%
Since $\xi$ is a random force, we can interpret this integral as a sum of random
steps of length $ \zeta^{-1} \xi(t') dt'$~\cite{doi_theory_88}. 

Let's write the integral as a sum over $N_t$ time steps of length $\delta
t$, \textit{i.e.}, $t = N_t \delta t$. This discretization gives
%
\begin{equation}
  x(t) - x(0) =  \int_0^t \zeta^{-1} \xi(t') \dif t' 
  \approx \sum_{i=1}^{N_t} \zeta^{-1} \xi_i \delta t.
\end{equation}
%
When we discretize time, we are assuming that the noise is constant over the
length of a time step. Since $\xi$ is uncorrelated in time, we are summing
uncorrelated steps just like in the discrete random walk. We can also translate
$\xi$ into a time discretization framework, $ \langle \xi_i \xi_j \rangle =
\frac{ 2 \zeta k_B T }{ \delta t} \delta_{ij}$~\cite{grassia_computer_95}.
Thus, the mean squared displacement can be written as
%
\begin{equation}
  \label{eqn:brownian_sum}
  \langle {(x(t) - x(0))}^2 \rangle 
  \approx \sum_{i=1}^{N_t} \sum_{j=1}^{N_t} 
  \zeta^{-2} \langle \xi_j \xi_j \rangle \delta t^2
  \approx = \frac{2 k_B T \delta t}{\zeta} N_t 
  =\frac{2 k_B T }{\zeta} t.
\end{equation}
%
Comparing \eqnref{eqn:brownian_sum} to \eqnref{eqn:msd}, we see the Einstein
relation~\cite{einstein_uber_05},
%
\begin{equation}
  \label{eqn:einstein}
  D =\frac{k_B T }{\zeta}.
\end{equation}
%
It is common to express the inverse friction as the mobility $\mu =
\frac{1}{\zeta}$.  I will mention the mobility throughout the thesis while
avoiding the use of the symbol $\mu$ for mobility to avoid confusion with the
chemical potential, also called $\mu$. \eqnref{eqn:einstein} can be derived
through other means~\cite{zwanzig_nonequilibrium_01, doi_theory_88}.  Typically,
we are looking at something a little more interesting: interacting particles,
external fields, etc.  Regardless, we imagine discretizing time and adding up
small positional changes that occurred over a time
interval~\cite{grassia_computer_95}.  The basic methods behind Brownian dynamic
simulations are summarized in \appxref{appx:brownian_dyn_sim}.

Finally, I connect the Brownian dynamics to a continuity
equation, \textit{i.e}, instead of examining a single random walk, what can we
say about the probability of finding the particle at a given location and time?
Since a Brownian particle undergoes a random walk, its position is a sum of
independent random variables.  By the central limit theorem, the distribution of
positions is Gaussian~\cite{doi_theory_88}.  The variance along any given
dimension $i$ must be $ \sigma_i^2 = \langle x_i^2(t) \rangle = 2Dt $.  Thus,
the probability density in $d$ dimensions is
%
\begin{equation}
  p(\bm{r},t) = 
  \frac{1}{{(4 \pi d D t)}^{d/2}} e ^ {-\frac{r^2}{4Dt}}.
\end{equation}
%
Here, I have assumed that the particle has started at origin; the $t
\rightarrow 0$ limit recovers a delta function centered at the origin.  This
distribution satisfies the diffusion equation
%
\begin{equation}
  \diff{p(\bm{r},t)}{t} = D \bm{\nabla} ^2 p(\bm{r},t).
\end{equation}
%
We are interested in tracking noninteracting Brownian particles.  The
single-particle probability distribution and concentration differ by a factor of
the particle number $N$ since there are no correlations between particles.  We
can therefore write the diffusion equation as
%
\begin{equation}
  \label{eqn:smoluchowski_diff}
  \diff{c(\bm{r},t)}{t}
  = -\bm{\nabla} \cdot \bm{j}_{\text{diff}}
  = \bm{\nabla} \cdot D \bm{\nabla} c(\bm{r},t).
\end{equation}
%
From this,  we can extract Fick's law of diffusion~\cite{doi_theory_88}
%
\begin{equation}
  \bm{j}_{\text{diff}} = -D \bm{\nabla} c(\bm{r},t).
\end{equation}
%
The diffusion coefficient is a constant scalar in our derivation because we
assumed that the friction coefficient is isotropic and homogeneous.  From here,
I briefly introduce obstacles and interactions.

%%%%%%%%%%%%% SECTION: EFFECTS OF OBSTACLES %%%%%%%%%%%%%
\section{Effects of obstacles}

While random walks are useful for describing dynamics of particle in many
situations, it is not complete.  Particle dynamics can by altered from
interactions with either other particles or the environment.  In crowded media,
obstacles impede tracer motion, \textit{e.g.}, diffusion of macromolecules in a
cell~\cite{hofling_anomalous_13}.  A profound effect of these obstacles is that
the random walk can become anomalous~\cite{metzler_random_00}.

One of the key assumptions in a normal random walk is that steps are attempted
at a constant frequency. When this occurs, the MSD is linear in time
\eqnrefp{eqn:msd}, which is referred to as Fickian or normal diffusion. If this
assumption is relaxed and the MSD is no longer linear in
time~\cite{metzler_random_00}, 
%
\begin{equation}
  \label{eqn:msd_anomal}
  \langle r^2 \rangle =  2dD t ^ {\alpha},
\end{equation}
%
the diffusion is said to be anomalous.  Randomly placed obstacles can
impede motion and cause a variable step frequency~\cite{sokolov_fractional_02,
  saxton_anomalous_94}.  In biological systems, anomalous diffusion has been
measured due to the presence of obstacles~\cite{feder_constrained_96,
  weiss_anomalous_04, ghosh_automated_94}.  Since the early work of
Saxton~\cite{saxton_lateral_87}, lattice Monte Carlo models have been an
important tool for studying diffusion in crowded environments.  Obstacles affect
the probability to move to a certain location.

While there are myriad lattice models for static obstacles with single-site hard
steric repulsion~\cite{saxton_lateral_87, saxton_anomalous_94}, multi-site hard
obstacles~\cite{ellery_characterizing_14, ellery_modeling_16}, and surface
binding~\cite{saxton_anomalous_96, ellery_analytical_16}, previous work did not
address bound motion and soft obstacles.  There are several biological systems
that suggest bound motion and soft obstacles, \textit{i.e.}, particles and
obstacles can occupy the same spatial region~\cite{brangwynne_germline_09,
  molliex_phase_15, hough_molecular_15,
  raveh_slideandexchange_16,timney_simple_16}.  The model we developed
in~\cite{stefferson_effects_17} studied how bound motion and binding affect
tracer dynamics. 

Another type of obstacle is a homogeneous distribution of binding sites that a
particle can attach to. The binding is controlled by binding kinetics
%
\begin{equation}
  T + B \rightleftharpoons C,
\end{equation}
%
where $T$ is the concentration of unbound particles, $B$ the concentration of
binding sites, and $C$ the concentration of the bound complex.  A diffusing
particle could bind to the uniform background of binding sites that alter its
motion. This was the model we developed to understand how biofilters select
which particles are allowed to pass through~\cite{maguire_design_18}.

%%%%%%%% SECTION: EFFECTS OF INTERACTIONS AND ACTIVITY  %%%%%
\section{Effects of interactions and activity}

Active matter is collections of interacting particles which are propelled by a
nonconservative force.  These novel systems illuminate how particle interactions
and activity can lead to new properties, such as giant number fluctuations and
unusual mechanical responses~\cite{marchetti_hydrodynamics_13}, that are not
present in equilibrium systems.  We are interested in how microscopic
interactions lead to bulk behavior.  Introducing interactions and activity to
the Langevin equation is straightforward.  In a simulation,  one can imagine
calculating forces on each particle, updating the positions, and analyzing the
results.  A single simulation, however, was just one possible
outcome, \textit{i.e.}, one realization of the noise.  To say something about
the average behavior, we would have to repeat the simulation many times.
Another method would be to evolve a probability density in time, like in the
diffusion equation \eqnrefp{eqn:smoluchowski_diff}, and I will refer to these as
continuity, continuous, or macroscopic equations.

When evolving the Langevin equation, the corresponding microscopic density of a
system of point-like particles is~\cite{hansen_theory_06}
%
\begin{equation}
  \label{eqn:density_operator}
  \hat{\rho}(\bm{r}) = \sum_{i=1}^N \delta( \bm{r} - \bm{r}_i ).
\end{equation}
%
This density is an operator since it is only truly meaningful in an integral.
The ensemble averaged density describes how the system behaves after averaging
over noise,
%
\begin{equation}
  \rho^{(1)}(\bm{r}) = \left\langle \hat{\rho}(\bm{r}) \right\rangle = 
  \left\langle \sum_{i=1}^N \delta( \bm{r} - \bm{r}_i
  ) \right\rangle,
\end{equation}
%
where $\rho^{(1)}(\bm{r})$ is the one-body density. Typically, the $(1)$
superscript is dropped or it is denoted as $c(\bm{r})$, and it is also referred
to as the density profile or concentration.  The one-body density is the noise
ensemble averaged probability of finding a particle at location $\bm{r}$. As $N$
increases, the BD numerics become unwieldy.  Solving continuous equations allows
us to evolve the dynamics at specified spatial coordinates instead of for all
$N$ particles.   

A technique for solving the temporal evolution of density is dynamical density
functional theory (DDFT)~\cite{marconi_dynamic_99, marconi_dynamic_00,
  archer_dynamical_04, archer_dynamical_05}.  Using DDFT, a continuity equation
for the evolution of the one-body density is expressed in terms of functional
derivatives  of the free energy $\mathcal{F}$
%
\begin{align}
  \label{eqn:ddft_intro}
  \diff{\rho(\bm{r,t})}{t} = 
  -\bm{\nabla} \cdot \bm{j} =
  \bm{\nabla}  \cdot \left[ \zeta^{-1} \rho(\bm{r},t) \bm{\nabla}
    \frac{\delta \mathcal{F} \left[ \rho(\bm{r}, t ) \right]}
  {\delta \rho(\bm{r}, t ) } \right].
\end{align}
%
We can interpret the DDFT flux $\bm{j}^{\mathcal{F}} = - \zeta^{-1} \rho(\bm{r},
t) \bm{\nabla} \frac{ \delta \mathcal{F} \left[ \rho \right]}{\delta \rho
  (\bm{r}, t) }$ as a concentration $ \rho(\bm{r},t ) $ times a velocity $
\bm{v} = - \zeta^{-1} \bm{\nabla} \frac{ \delta \mathcal{F} \left[ \rho \right]
} {\delta \rho (\bm{r}, t ) } $. The velocity term arises from the gradients of
chemical potential $ \mu( \bm{r} ) = \frac{ \delta\mathcal{F} \left[ \rho
  \right] } {\delta \rho (\bm{r} )} $.

The nonconservative active force cannot be expressed as a gradient of a chemical
potential.  However, it can described by an active flux $ \bm{j}^{D} =
\rho(\bm{r},t)\bm{v}^{\text{D}} $.  Our total flux will therefore be given by
%
\begin{equation}
  \bm{j} = \bm{j}^{\mathcal{F}} + \bm{j}^{\text{D}}.
\end{equation}
%
We can separate the free energy into contributions from ideal gas entropy
$\mathcal{F}^{\text{id}}$, interactions or excess $\mathcal{F}^{\text{ex}}$, and
external fields $\mathcal{F}^{\text{ext}}$~\cite{hansen_theory_06}, 
%
\begin{equation}
  \mathcal{F} = \mathcal{F}^{\tx{id}} + \mathcal{F}^{\tx{ex}} 
  +\mathcal{F}^{\tx{ext}}.
\end{equation}
%
The ideal gas and external field contributions are known exactly. The crux of
DDFT is approximating a functional for the
interactions~\cite{archer_dynamical_05}.

In this thesis, I will describe how the DDFT framework can be used to solve for
the temporal evolution of active systems. I will then apply it to a
self-propelled hard needle
system~\cite{baskaran_hydrodynamics_08,baskaran_enhanced_08,
  wensink_aggregation_08, baskaran_selfregulation_12, ginelli_largescale_10,
  kuan_hysteresis_15}. I will present novel results including a banding
instability and discuss where my work fits into the larger body of
work~\cite{stefferson_selforganized_18}.
